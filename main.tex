\documentclass[12pt,a4paper]{report}

\usepackage{array}
\usepackage{makecell}
\newcolumntype{L}[1]{>{\raggedright\let\newline\\\arraybackslash\hspace{0pt}}m{#1}}
\newcolumntype{C}[1]{>{\centering\let\newline\\\arraybackslash\hspace{0pt}}m{#1}}
\newcolumntype{R}[1]{>{\raggedleft\let\newline\\\arraybackslash\hspace{0pt}}m{#1}}

\usepackage{mathptmx}
\usepackage{graphicx} % Required for inserting images
\graphicspath{{img/}}
\usepackage{multirow}
\usepackage{caption}
\usepackage{longtable}

\usepackage{geometry}
\geometry{
	top=4cm,
	right=3cm,
	left=4cm,
	bottom=3cm,	
}

\usepackage{pdflscape}
\usepackage{tabularray}
\usepackage[table]{xcolor}
\usepackage{setspace}
\onehalfspacing
  
\renewcommand{\thechapter}{\centering \Roman{chapter}}
\renewcommand{\thesection}{\arabic{chapter}.\arabic{section}}

\def\contentsname{DAFTAR ISI}
\renewcommand\bibname{DAFTAR PUSTAKA}
\def\chaptername{BAB}

\usepackage{titlesec}
\titleformat{\chapter}[display]{\normalfont\bfseries\centering}{\MakeUppercase{\chaptertitlename}~\thechapter}{0pt}{}
\titlespacing*{\chapter}{0pt}{-2pt}{16pt}

\renewcommand{\arraystretch}{1.5}

\renewcommand\thechapter{\Roman{chapter}}
\renewcommand\thesection{\arabic{section}}
\def\thesection{\arabic{chapter}.\arabic{section}}
\def\thetable{\arabic{table}}

\titleformat{\section}[block]{\bf\normalsize}{\thesection}{0.6em}{}
\titlespacing*{\section}{0pt}{5pt}{0pt}
\titleformat{\subsection}[block]{\bf\normalsize}{\thesubsection}{0.6em}{}
\titlespacing*{\subsection}{0pt}{5pt}{0pt}

\usepackage{setspace}
%\singlespacing
\onehalfspacing
%\doublespacing
%\setstretch{1.1}
\renewcommand{\tablename}{Tabel}
\renewcommand{\figurename}{Gambar}
\renewcommand{\thefigure}{\thesection.\arabic{figure}}
\renewcommand{\thetable}{\thesection.\arabic{table}}
\usepackage[breaklinks]{hyperref}
\renewcommand{\listfigurename}{DAFTAR GAMBAR}
\renewcommand{\listtablename}{DAFTAR TABEL}

\hyphenation{me-la-in-kan}

\newcommand{\nim}{1214434}
\newcommand{\mahasiswa}{Muhammad Davi}
\newcommand{\judulId}{Penulisan Skripsi}
\newcommand{\judulEn}{Thesis Title}
\newcommand{\jurusan}{Jurusan Teknologi Informasi dan Komputer}
\newcommand{\prodi}{Teknologi Rekayasa Komputer Jaringan}
\newcommand{\institusi}{Politeknik Negeri Lhokseumawe}
\newcommand{\pembimbingUtama}{Muhammad Davi}
\newcommand{\pembimbingPendamping}{Muhammad Reza}

\begin{document}

\begin{titlepage}
\begin{center}

\MakeUppercase{skripsi}

\vfill
\begin{figure}[h]
\centering
\includegraphics[width=2.54cm]{logo-pnl}
\end{figure}

\vfill
\MakeUppercase{\judulId}

\vfill
Oleh: \\
\MakeUppercase{\mahasiswa} \\
\nim \\

\vfill
\MakeUppercase{
program studi \prodi \\
\jurusan \\
\institusi \\
\the\year{}
}

\end{center}
\end{titlepage}
\pagenumbering{roman}
\begin{titlepage}
\begin{center}
LEMBAR PENGESAHAN PROPOSAL TUGAS AKHIR

\vspace*{1cm}
\begin{tabular}{llp{10cm}}
Judul Tugas Akhir	& : & \MakeUppercase{\judulId} \\
Nama Mahasiswa 		& : & \MakeUppercase{\mahasiswa} \\
NIM 				& : & \nim \\
Program Studi 		& : & \prodi \\
\end{tabular}

\vspace*{1cm}
Menyetujui: \\
Pembimbing I

\vspace*{2cm}
\pembimbingUtama \\
NIP: \nipPembimbingUtama
	
\vspace*{1cm}
Pembimbing II

\vspace*{2cm}
\pembimbingPendamping \\
NIP: \nipPembimbingPendamping

\vfill
Mengetahui \\
Ka. Prodi \prodi

\vspace*{2cm}
\kaprodi \\
NIP: \nipKaprodi
\end{center}
\end{titlepage}

\tableofcontents
\addcontentsline{toc}{chapter}{DAFTAR ISI}


\listoftables
\listoffigures


\chapter*{RINGKASAN}
\addcontentsline{toc}{chapter}{RINGKASAN}
\noindent

Kemajuan perkembangan teknologi game yang tiap 
tahunnya berkembang pada perangkat lunak dan perangkat keras 
khususnya pada perangkat mobile yang mendukung sistem 
operasi Android, kini telah mumpuni menjalankan aplikasi game 
berskala 3D dan pengintegrasian antar perangkat lunak dan 
perangkat keras yang menghasilkan performa yang optimal. 

Pengalaman bermain pengguna sangat bervariasi dari 
segi aspek yang berbeda-beda cara untuk meningkatkan 
pengalaman bermain pada pengguna. Bermain dalam satu waktu 
yang sama dalam sebuah permainan dengan pemain lainnya 
merupakan salah satu caranya untuk menciptakan pengalaman 
bermain game menjadi lebih menyenangkan. 

Dalam Tugas Akhir ini dibangun sebuah permainan yang 
bergenre fsp yang memiliki gameplay yang interaktif untuk 
mengendalikan karakter dan menerapkan tipe permainan 
synchronous multiplayer untuk mode online dalam sebuah 
permainan. Antar pemain dapat langsung berinteraksi ketika 
bermain karena dipertemukan secara realtime. Metode 
penerapan untuk merealisasikannya tersebut akan menggunakan 
perpaduan dari unity engine dengan framework photon unity 
network untuk mengintegrasikan unity engine dengan photon 
cloud. Dengan fungsionalitas aplikasi permainan tersebut 
diketahui bahwa cara bermain yang mempertemukan antar 
pemain secara realtime akan menciptakan suasana 
menyenangkan dalam bermain.

\noindent \textbf{Kata Kunci: FPS Multiplayer, FPS Game, Unity, Photon Unity, Photon Cloud, Photon Unity Networking.}
\clearpage

\pagenumbering{arabic}
\setcounter{page}{1}

\chapter{PENDAHULUAN}
\section{Latar Belakang Masalah}
\noindent

\textit{Game} atau permainan adalah aktivitas yang dilakukan untuk tujuan hiburan atau kompetisi, dengan aturan yang telah ditentukan dan biasanya memiliki elemen interaktif yang melibatkan satu atau lebih peserta. \textit{Game} sering kali melibatkan strategi, kecepatan, keterampilan, atau ketangkasan fisik, tergantung pada jenisnya. Tujuan dari \textit{game} adalah untuk mencapai kemenangan, skor tinggi, atau hanya untuk kesenangan semata. \textit{Game} bisa dimainkan secara individu atau dalam kelompok, dan dapat berupa permainan fisik seperti sepak bola, permainan papan seperti catur atau permainan video seperti Mario Bros\cite{fps}.

\textit{First Person Shooter} merupakan sebuah permainan peperangan menggunakan senjata api dengan sudut pandang orang pertama dan hanya menampilkan senjata yang digunakan.
Dalam permainan FPS, pemain biasanya melawan musuh secara langsung dalam pertempuran yang cepat dan intens\cite{fps}. Senjata api menjadi alat utama pemain dalam memerangi musuh.
Agar \textit{game} \textit{First Person Shooter} (FPS) lebih menarik dimainkan, peniliti menambahkan fitur \textit{multiplayer} agar dapat dimainkan bersama sama secara online yang dapat terhubung dimana saja dengan menggunakan koneksi internet. Untuk membuat fitur \textit{multiplayer} peniliti menggunakan \textit{game} engine unity dan framework photon unity networking.

Sistem \textit{multiplayer} pada sebuah \textit{\textit{game}} membuat \textit{\textit{game}} tersebut menjadi lebih interaktif dan menarik untuk dimainkan. Dalam sebuah gim jika pemain memilih untuk single player maka pemain tersebut akan berhadapan dengan lawan NPC (Non Playable Character) sedangkan jika \textit{multiplayer} maka pemain tersebut akan berhadapan dengan pemain lain \cite{Sarwodi}.


Penelitian ini mengusulkan sebuah \textit{\textit{game}} dengan memanfaatkan koneksi via internet yang dapat memainkan \textit{\textit{game}} bertema \textit{first person shooter}, dimana pemain bersaing secara real (nyata) dan lebih menantang di mana minimal ada 2 pemain yang akan bertemu dalam satu room.
Berdasarkan penjabaran diatas, maka diusulkan sebuah judul skripsi yang mengimplementasikan koneksi internet menggunakan photon unity asset pada \textit{\textit{game}} \textit{first person shooter} 3D yang dapat dimainkan menggunakan perangkat windows dengan judul "\judulId".
\textit{\textit{game}} ini akan dibuat \textit{multiplayer} menggunakan fitur dari unity \textit{\textit{game}} engine yaitu photon unity networking.

\section{Rumusan Masalah}
\noindent

Berdasarkan latar belakang masalah yang telah diuraikan, maka didapat perumusan masalah sebagai berikut :
\begin{enumerate}
	\item Bagaimana merancang skenario game fps jak meuprang ?
	\item Bagaimana cara kerja fitur sinkronisasi \textit{multiplayer} online secara \textit{realtime} dengan menggunakan Photon Unity Networking dan data apa saja yang perlu disinkronisasi?
	\item Bagaimana mekanisme alur kerja \textit{first person shooter}(fps) \textit{multiplayer} dari proses persiapan bermain, mulai bermain sampai menyelesaikan permainan?
	\item Berapa persen tingkat keberhasilan pengukuran performa jaringan pada saat game dimainkan?
\end{enumerate}	
\section{Tujuan Penelitian}
\noindent

Adapun tujuan dari penelitian ini sebagai berikut :
\begin{enumerate}
	\item Untuk menghasilkan skenario game \textit{deathmatch} pada game fps jak meuprang.
	\item Untuk mengetahui apa saja yang disinkronisasikan pada game tersebut.
	\item Untuk mengetahui \textit{gameplay} game jak meuprang.
	\item Untuk mengetahui ke-stabilan kinerja jaringan, latensi, penggunaan bandwidth, dan lainnya. 
\end{enumerate}


\section{Batasan Masalah}
\noindent

Pada penelitian ini terdapat batasan masalah dengan maksud untuk mempermudah penulis, adapun batasan masalah pada penelitian ini sebagai berikut:
\begin{enumerate}
	\item Pembuatan \textit{\textit{game}} ini akan menggunakan IDE Unity dan bahasa pemrograman C\#.
	\item Total maksimum CCU (\textit{Concurent Users}) yang dapat terhubung ke Photon Cloud yaitu 5 CCU.
	\item Hanya dapat dimainkan diplatform windows.
	\item Hanya dapat dimainkan jika perangkat terhubung dengan koneksi internet.
	\item Tersedia sound.
	\item Tersedia senjata sebanyak 3 jenis yaitu rifle, pistol dan pisau.
	\item Menggunakan assets open source.
	\item Tersedia Map.
	\item Terdapat dua karakter.
\end{enumerate}

\section{Manfaat Penelitian}
Manfaat dari penilitian ini antara lain adalah : 
\begin{enumerate}
	\item Memberikan hiburan dan melatih ketangkasan bermain 
	kepada pengguna.
	\item Untuk mengetahui performa jaringan photon cloud yang dimiliki photon unity networking.
	\item Sebagai bentuk implementasi konsep photon unity networking pada \textit{\textit{game}} first person shooter(fps).
\end{enumerate}
\begin{landscape}
	\chapter{TINJAUAN PUSTAKA}
	\section{\textit{State of the Art}}
	\noindent

	\textit{State of the Art} Dalam penyusunan penilitian ini, peniliti mengambil beberapa referensi terdahulu sebagai panduan penulis untuk penilitian yang dilakukan, yang kemudian  akan menjadi acuan dan perbedaan dari penilitian yang akan dilakukan dengan penilitian sebelumnya. Pemaparan \textit{State of the Art} dapat dilihat pada tabel \ref{tb:stateoftheart} berikut.
	
	\begin{center}
	\begin{longtable}{| c | L{3cm} | L{4cm} | L{2.5cm} | L{4cm} | L{3cm} | L{3cm} |}
	\caption{Paparan \textit{State of the Art}}
	\label{tb:stateoftheart} \\
	
	\hline 
	No &
	Penulis/Tahun &
	\multicolumn{1}{c|}{Judul Artikel} &
	\multicolumn{1}{c|}{Metode yang digunakan} &
	\multicolumn{1}{c|}{Hasil yang diperoleh} &
	\multicolumn{1}{c|}{Persamaan} &
	\multicolumn{1}{c|}{Perbedaan} \\ \hline
	\endfirsthead
	
	\hline 
	No &
	Penulis/Tahun &
	\multicolumn{1}{c|}{Judul Artikel} &
	\multicolumn{1}{c|}{Metode yang digunakan} &
	\multicolumn{1}{c|}{Hasil yang diperoleh} &
	\multicolumn{1}{c|}{Persamaan} &
	\multicolumn{1}{c|}{Perbedaan} \\ \hline
	\endhead
	
	\hline \multicolumn{7}{|r|}{{Bersambung}} \\ \hline
	\endfoot
	
	\hline \hline
	\endlastfoot
	1 	& Ibnu Ramadhan, Agung Purwanto dan Nurahman (2020) \cite{fps}
	& PENGEMBANGAN TEKNOLOGI \textit{GAME} INDONESIA UNTUK PERMAINAN \textit{FIRST PERSON SHOOTER} (FPS) 3D \textit{MULTIPLAYER} “CODE TO SHOOT” MENGGUNAKAN UNITY NETWORK (UNET) BERBASIS MOBILE
	& Unity Network
	& \textit{Game} ini dapat dimainkan secara \textit{multiplayer} tanpa perlu memasukan alamat IP karena fitur uNet dapat bekerja dengan baik. Selain itu \textit{game} ini juga sudah dapat dimainkan menggunakan platform mobile (Android)
	& Sama sama \textit{game} bergenre \textit{first person shooter}(fps)
	& Perbedaannya peniliti menggunakan unet sebagai \textit{multiplayer} platform
	\\ \hline
	2 	& Shena Star Sarwodi, Wibisono Sukmo Wardhono, Muhammad Aminul Akbar (2020) \cite{Sarwodi}
	& Penerapan \textit{Multiplayer} Pada Gim Tower Defense Menggunakan Photon Unity
	& Photon Unity Networking
	& Dengan menerapkan Photon Unity Networking pada gim tower defense maka dapat diimplementasikan sebuah fitur yang dapat meningkatkan interaktivitas dan ketertarikan pemain pada gim yaitu fitur \textit{multiplayer}
	&Persamaannya yaitu menggunakan unity photon.
	&Perbedaannya terdapat pada game yang diterapkan .
	\\ \hline

	3 	& Ryan Nanda Pratama,  Anton Siswo Raharjo Ansori, Ashri Dinimaharawati (2021) \cite{Ansori}
		& PEMBUATAN \textit{MULTIPLAYER} \textit{GAME} UCING BELING MENGGUNAKAN ASSET STORE MIRROR
		& Asset Mirror
		& Pada \textit{multiplayer} \textit{game} Ucing Beling dapat dimainkan secara realtime dan berjalan dengan sesuai yang diharapkan
		& Persamaannya sama sama \textit{multiplayer}.
		& Perbedaannya peniliti menggunakan asset store mirror sebagai \textit{multiplayer} platform
		\\ \hline	
	
		4 	& I Kadek Budi Suartama, I Gede Mahendra Darmawiguna, dan I Made Putrama (2020) \cite{Gebug}
		& PENGEMBANGAN GAME MULTIPLAYER PENGENALAN BUDAYA GEBUG ENDE SERAYA KARANGASEM BERBASIS ANDROID
		 & Metode pengembangan dalam penelitian ini menggunakan GDLC (Game Development Life Cycle)
		 & pengujian blackbox mendapatkan hasil bahwa semua fungsi dan fitur-fitur yang ada dapat 
		 berjalan dengan baik dan sebagaimana mestinya.
		 & Persamaannya yaitu menggunakan game engine unity.
		 & Perbedaannya terdapat pada metode yang digunakan.
		 \\ \hline

	
	
	5 	& Muhammad Faisal Fathurrohman Dan Iskandar Ikbal (2018) \cite{gogreen}
		& PEMBANGUNAN GAME MULTIPLAYER EDUKASI GO GREEN 3D BERBASIS ANDROID 
		& Google Play Games Realtime Multiplayer
		& Dapat terhubung secara multiplayer menggunakan google play games realtime multiplayer.
		&Persamaannya yaitu sama sama base multiplayer.
		&Perbedaannya terdapat pada google play games realtime multiplayer.
		\\ \hline
			  
	\end{longtable}
	\end{center}
	\end{landscape}

\section{Tinjauan Teoritis}
\subsection{Unity}
\noindent

Unity merupakan salah satu \textit{game} engine paling populer saat ini. Penggunaan Unity dapat digunakan untuk mengembangkan konten interaktif seperti video \textit{game}, 
visualisasi arsitektur, dan real-time 3D animasi. Unity menggunakan bahasa pemograman JavaScript dan 
C\# \cite{Ansori}. Unity juga merupakan perangkat lunak yang digunakan untuk mengembangkan \textit{game} \textit{multiplatform} yang didesain secara user \textit{friendly} 
(Iman, 2017). Keunggulan Unity adalah Unity 
dapat dengan mudah mengontrol objek-objek 
dalam gim atau aplikasi. Unity terdapat 2 jenis 
lisensi yaitu \textit{personal edition} yang dapat diakses 
secara gratis dan \textit{professional edition} yang 
diharuskan untuk membayar perbulan untuk 
mengaksesnya dengan beberapa fitur tambahan 
yang tidak terdapat di \textit{personal edition} \cite{Sarwodi}. 

\subsection{\textit{Multiplayer}}
\noindent

\textit{Multiplayer} merupakan fitur pada \textit{game} dimana pemain bermain dengan lebih dari 1 orang yang bermain 
di lingkungan \textit{game} yang sama dan waktu yang bersamaan. \textit{Game} \textit{Multiplayer} biasanya memberikan pilihan pada 
pemain untuk berbagi sumber daya sistem \textit{game} atau menggunakan internet untuk bermain bersama dalam jarak 
jauh. \textit{Game} \textit{Multiplayer} yang terhubung dengan internet melibatkan pemain yang saling terhubung melalui server. 
Sedangkan \textit{Game} \textit{Multiplayer} dengan koneksi lokal yaitu, pemain saling terhubung secara langsung dengan 
pemain lainnya, pemain terkoneksi menggunakan jaringan peer to peer. Pada \textit{Game} \textit{Multiplayer} online memiliki 
beberapa jenis kategori diantaranya adalah \textit{Massively} \textit{Multiplayer} Online \textit{game} (MMO), \textit{Massively} \textit{Multiplayer} 
Online \textit{First-person Shooter} \textit{Game} (MMOFPS), \textit{Massively} \textit{Multiplayer} online \textit{Real-time Strategy} \textit{Game}
(MMORTS), \textit{Massively} \textit{Multiplayer} Online Role-playing \textit{Game}s (MMORPG), \textit{Multiplayer} Online Battle Arena
(MOBA)\cite{Ansori}. 

\subsection{Photon Unity Networking (PUN)}
\noindent

Photon adalah sebuah framework pengembangan \textit{game} \textit{multiplayer} \textit{real-time} yang cepat, ringan, dan fleksibel. Photon terdiri dari server dan beberapa SDK klien untuk platform utama.
Photon Unity Network (PUN) adalah solusi khusus Unity yang dihadirkan dengan tingkat yang lebih tinggi: matchmaking, panggilan balik yang mudah digunakan, komponen untuk sinkronisasi \textit{Game}Objects, Remote Procedure Calls (RPCs), dan fitur serupa yang memberikan awal yang baik. Di luar itu, terdapat API yang solid dan luas untuk kontrol yang lebih canggih \cite{pun}.
Berikut gambaran 
integrasi aplikasi dengan Photon Unity Networking pada Gambar \ref{fig:photonni}
\begin{figure}[ht]
	\centering
	\includegraphics[width=10cm]{arsitektur-photon.png}
	\caption{Fitur Photon Unity Networking}
	\label{fig:photonni}
\end{figure}
\subsection{\textit{First Person Shooter}(FPS)}
\noindent

\textit{First Person Shooter} (FPS) adalah salah satu jenis \textit{game} yang saat ini sangat digemari terutama kalangan \textit{game}rs muda. FPS merupakan \textit{game} yang menggunakan sudut pandang orang pertama dimana pemain akan dibuat seolah-olah menjadi karakter utama dalam \textit{game} dengan tampilan yang berpusat pada permainan disekitar senjata atau alat yang sedang digunakan \cite{fps}.

\textit{First person shooter} merupakan jenis 3D \textit{game} shooter yang menampilkan sudut pandang orang pertama dengan 
pemain yang melihat aksi melalui mata karakter permain. Tidak seperti orang ketiga yang terlihat dari bagian 
belakang atau samping, yang memungkinkan \textit{game}r untuk melihat karakter secara keseluruhan\cite{fps}.

FPS dikembangkan pada tahun 1973 melalui permainan ruang yang belum sempurna yaitu flight simulator, yang 
menampilkan sudut pandang orang pertama dengan mengarah lebih rinci ke simulator pesawat tempur, dikembangkan untuk pasukan AS pada akhir tahun 1970-an. Permainan ini tidak lagi tersedia untuk konsumen \cite{fps}.

\subsection{C\#}
\noindent

C\# (C-sharp) adalah salah satu bahasa pemograman yang menggunakan Framework .NET. Sama seperti 
bahasa lainnya, C\# memiliki aturan pada syntax dan kode-kode yang bisa digunakan dalam pembuatan aplikasi. 
C\# cocok untuk dipelajari untuk pemula karena aturan syntax-nya lebih sederhana dibandingkan bahasa 
pemograman lainnya \cite{Ansori}.

\subsection{Wireshark}
\noindent

Wireshark merupakan sebuah \textit{software} penganalisa jaringan yang paling dikenal. \textit{Software} ini 
sangat berguna dalam menyediakan jaringan dan protokol serta memberikan informasi tentang 
data yang tertangkap pada jaringan. Software wireshark dapat menganalisa transmisi paket data 
dalam jaringan, proses koneksi dan transmisi data antar komputer\cite{wireshark}.

\subsection{\textit{Quality Of Service (QoS)}}
\noindent

QoS (Quality Of Service) adalah parameter-parameter yang menjadi indicator bagus atau
tidaknya performansi dari suatu jaringan. Parameter
yang menjadi indikator dalam QoS ini meliputi 
Bandwidth, Troughput, dan \textit{Packet Loss}, delay, dan
jitter \cite{qos}. Untuk itu dilakukan analisis Quality Of
Service (QoS) pada jaringan photon cloud .Adapun 
standar pengkuran performansi dalam suatu jaringan 
yaitu TIPHON (Telecommunicationsand Internet 
Protocol Harmonization Over Networks) yang 
mengkategorikan beberapa performansi dalam 
perhintungan tertentu.

\begin{enumerate}
	\item \textit{\textit{Throughput}} \\
	\textit{Throughput} merupakan parameter QoS yang 
menunjukkan suatu kecepatan rata-rata bandwidth
yang sebenarnya, diukur dengan satuan waktu 
tertentu pada kondisi jaringan tertentu untuk 
melakukan pengiriman paket dengan ukuran tertentu 
juga. Hasil \textit{throughput} diambil dari jumlah paket 
data yang dikirim dibagi dengan jumlah waktu yang 
diperlukan saat pengiriman paket data.
	\item \textit{Packet Loss} \\
	\textit{Packet Loss} merupakan suatu parameter QoS 
yang menunjukkan suatu jumlah total keseluruhan 
paket hilang atau tidak sampai ke destinasi, 
dikarenakan adanya overload atau congestion pada 
jaringan. Dalam suatu jaringan, \textit{packet loss}
diwajibkan mempunyai persentase yang kecil sesuai 
dengan standar. 
\item \textit{Delay} \\
\textit{Delay} merupakan suatu parameter QoS yang 
menunjukkan jumlah waktu yang diperlukan paket 
untuk mencapai jarak dari source ke destination. 
Berberapa hal yang mempengaruhi delay adalah 
jarak, perangkat keras dan congestion. 
\item \textit{Jitter}\\
\textit{Jitter} merupakan suatu parameter QoS yang 
menunjukkan jumlah dari variasi-variasi delay pada 
transmisi paket pada jaringan. Hal ini disebabkan 
banyaknya variasi panjang antrian paket dalam 
waktu proses paket dan waktu penghimpunan ulang 
paket-paket.


\end{enumerate}






\chapter{METODOLOGI PENELITIAN}
\noindent

Pada bagian ini peniliti menggunakan dua jenis data yaitu data primer dan data sekunder. Data primer adalah data yang diperoleh dari hasil kuisoner dan data sekunder yang diperoleh dari hasil penilitian sebelumnya.

\section{Data dan Pengumpulan Data}
\noindent

Penulis menggunakan beberapa tahap atau metode dalam melakukan penilitian untuk menyusul proposal skripsi, yaitu :

\begin{enumerate}
    \item Studi Pustaka \\ Peneliti mengumpulkan data dengan cara mencari dari internet dan jurnal yang menyangkut atau jurnal yang membahas photon unity networking dalam pembuatan game multiplayer.
    \item Observasi \\ Peniliti mengumpulkan data dengan cara memainkan sekaligus mengamati secara langsung permainan sejenis yang sudah ada.
\end{enumerate}

\section{Rancangan Sistem(software/hardware)}
\noindent

    Pada penilitian ini membutuhkan \textit{software} dan perangkat keras untuk melakukan pembuatan game first person shooter. Berikut ini spesifikasi rancangan sistem penilitian yang dijabarkan pada Tabel \ref{tb:tabel-spesifikasi}
    
    \begin{table}[h]
        \centering
        \begin{tabular}{|ll|lll}
        \cline{1-2}
        \multicolumn{2}{|c|}{Software}                                                &  &  &  \\ \cline{1-2}
        \multicolumn{1}{|l|}{Sistem Operasi} & Windows 11                             &  &  &  \\ \cline{1-2}
        \multicolumn{1}{|l|}{Tools}          & Unity 3D                                 &  &  &  \\ \cline{1-2}
        \multicolumn{2}{|c|}{Perangkat Keras}                                         &  &  &  \\ \cline{1-2}
        \multicolumn{1}{|l|}{Processor}      & Amd Ryzen 5 5400H With Radeon Graphics &  &  &  \\ \cline{1-2}
        \multicolumn{1}{|l|}{Memory}         & 8192MB Ram                             &  &  &  \\ \cline{1-2}
        \multicolumn{1}{|l|}{Video Card}     & Nvidia GeForce RTX 3050                &  &  &  \\ \cline{1-2}
        \multicolumn{1}{|l|}{SSD}            & 460GB                                  &  &  &  \\ \cline{1-2}
        \end{tabular}
        \caption{Tabel spesifikasi}
        \label{tb:tabel-spesifikasi}
        \end{table}

% \subsection{Rancangan Algoritma Matchmaking}
% Matchmaking dalam fitur permainan ini berperan sangat penting dan merupakan peran utama. Dalam peran tersebut, matchmaking mempertemukan antar pemain yang sedang aktif dalam melakukan pencarian room. Berikut penjelasan algoritma dalam bentuk \textit{flowchart} pada gambar 1.
% \begin{figure}[h]
%     \centering
%     \caption{Konteks  Algoritma Matchmaking}
%     \includegraphics[width=15cm]{flowchart-matchmaking}
%     \label{fig:algoritmamatmaching}
%     \end{figure}


%     Mengacu pada gambar \ref{fig:algoritmamatmaching}, algoritma membutuhkan daftar \textit{room} yang didapatkan dari \textit{interface} photon \textit{Behavior}. Jika belum ada \textit{room} yang tersedia pada daftar \textit{room}, maka dilakukan fungsi detil dari algoritma matchmaking yaitu SearchRoomMatchmaking(). Berikut detil penjelasan flowchart pada gambar \ref{fig:detail-mm-1} dan gambar \ref{fig:detail-mm-2}.
%     \newpage
%     \begin{figure}[h]
%         \centering
%         \caption{Detail Algoritma Matchmaking 1}
%         \includegraphics[width=10cm]{detail-algoritma-matmaking1.png}
%         \label{fig:detail-mm-1}
%         \end{figure}


%         Mengacu pada Gambar \ref{fig:detail-mm-1}, ketika fungsi ini dipanggil, 
% maka ia akan memulai perulangan sebanyak daftar room yang 
% tersedia. Pencarian bertujuan untuk mencari room yang memiliki 
% status sedang tidak bermain (waiting). Jika tidak menemukan 
% room yang tersedia dan ber-status tidak sedang bermain (waiting) 
% maka akan dilakukan pemanggilan fungsi 
% RealocateRoomMatchmaking() yang dijelaskan lebih detil pada 
% Gambar \ref{fig:detail-mm-2}. Jika sebaliknya maka pemain akan bergabung ke 
% dalam room.

% \newpage
% \begin{figure}[h]
%     \centering
%     \caption{Detail Algoritma Matchmaking 2}
%     \includegraphics[width=10cm]{detail-algoritma-matmaking2.png}
%     \label{fig:detail-mm-2}
%     \end{figure}

\subsection{Rancangan Use Case Diagram}
\begin{enumerate}
    \item \textit{Use Case Diagram}
    \\ Tahapan ini memiliki satu aktor yaitu pemain, penjelasan mengenai tahap ini diilustrasikan pada gambar \ref{fig:case-diagram}
    \begin{figure}[h]
        \centering
        \includegraphics[width=10cm]{case-diagram.png}
        \caption{Use Case Diagram}
        \label{fig:case-diagram}
    \end{figure}

    Gambar \ref{fig:case-diagram} menjelaskan tentang \textit{Use Case Diagram} dimana terdapat satu aktor yaitu pemain, serta 5 \textit{Use Case} yaitu \textit{Buat Room, Join Room, Exit Room, About Game, Exit Game}.
    Pemain dapat menjadi server jika pemain melakukan \textit{use case} buat room dan pemain juga bisa menjadi client jika pemain menggunakan \textit{use case join room}, tetapi jika tidak adanya pemain yang menggunakan \textit{use case} buat room, maka \textit{use case} join room tidak dapat digunakan oleh pemain yang menjadi client.
    \textit{use case exit room} dapat dilakukan oleh pemain yang menjadi server maupun menjadi client, dan hal itu akan menghapus sesi room yang sudah dibuat jika semua pemain menggunakan \textit{use case exit room}. \textit{Use case about game} dapat dilakukan oleh pemain untuk mengetahui tentang game yang dimainkan, dan yang terakhir jika pemain ingin meninggalkam game, pemain mengunakan \textit{Use case Exit Game.}
    \textit{Activity diagram} adalah diagram yang memodelkan aliran aktivitas pada sistem.
    \item Activity Diagram
    \begin{enumerate}
        \item \textit{Activity Diagram} buat \textit{Room}
        \begin{figure}[h]
           \centering
           \includegraphics[width=10cm]{room-diagram.png}
           \caption{Activity Diagram Create Room}
           \label{fig:croom-case}
       \end{figure}
       \\ Gambar \ref{fig:croom-case} merupakan \textit{Activity Diagram Create Room}. Diawali dengan pemain membuka game, sistem akan menampilkan menu game, player menekan tombol buat room untuk membuat room yang belum tersedia sebagai server atau pemilk room.
       Pada \textit{activity waiting player} pemilik room akan menunggu player/\textit{client} untuk memasukan room dan memulai game.
       \newpage
    \item \textit{Activty Diagram} join \textit{Room}
    \begin{figure}[h]
        \centering
        \includegraphics[width=10cm]{joinroom-diagram.png}
        \caption{Activity Diagram Join Room}
        \label{fig:jroom-case}
    \end{figure}
    \\ Gambar \ref{fig:jroom-case} merupakan \textit{activity diagram join room}. Diawali dengan pemain membuka game, kemudian sistem akan menampilkan menu dari game, player menekan tombol join room, photon unity akan mencarikan room yang tersedia yang sudah dibuat oleh pemain lain atau bisa mengetikan manual custom port yang terdapat pada pembuat room. Pemain akan terhubung ke server atau room yang tersedia jika room valid.
    \newpage
    \item  \textit{Activity Diagram Exit Room}
    \begin{figure}[h]
        \centering
        \includegraphics[width=10cm]{exitroom-diagram.png}
        \caption{Activity Diagram Exit Room}
        \label{fig:eroom-case}
    \end{figure}
    \\ Gambar \ref{fig:eroom-case} merupakan \textit{activity diagram exit room}. Diawali dengan pemain menekan tombol exit room, sistem akan menampilkan menu utama pada game.    
    \item \textit{Activity Diagram About Room}
     \begin{figure}[h]
        \centering
        \includegraphics[width=10cm]{aboutcase-diagram.png}
        \caption{Activity Diagram About Room}
        \label{fig:aroom-case}
    \end{figure}
    \\ Gambar \ref{fig:aroom-case} merupakan \textit{activity diagram about room}. Diawali dengan pemain menekan tombol \textit{about room}. Sistem akan menampulkan isi tentang game yang dimainkan.
    \item  \textit{Activity Diagram Exit Game}
    \begin{figure}[h]
        \centering
        \includegraphics[width=10cm]{exitgame-diagram.png}
        \caption{Activity Diagram Exit Game}
        \label{fig:egame-case}
    \end{figure}
    \\ Gambar \ref{fig:egame-case} merupakan \textit{activity diagram exit game}. Diawali dengan pemain menekan tombol \textit{exit game}. Sistem akan mengeluarkan pemain dari game yang sedang dimainkan.
\end{enumerate}
\end{enumerate}

    % \item \textit{Activity Player}
    % \begin{figure}[h]
    %     \centering
    %     \includegraphics[width=10cm]{player-case.png}
    %     \caption{Activity Diagram Player}
    %     \label{fig:player-case}
    % \end{figure}
    % \\ Gambar \ref{fig:player-case} merupakan \textit{Activity Diagram Player}. Diawali dengan pemain harus terkoneksi jaringan photon, kemudian sistem akan memasuki ke layer network, yang akan menghubungkan beberapa komponen yaitu player movement, player health dan player info.
    % Pada \textit{photon RPC} player memasuki ke sistem sinkronisasi kedalam \textit{projectileshoot} yang berfungsi disini sebagai menambak player musuh, kemudian memasukin activit weapon info yang berisi \textit{ammo weapon}, dan dari ammo yang tersedia mamsuki activity playerfire yang menandakan player siap memerangi player lain.
    % \newpage
    



% \subsection{Rancangan Arsitektur Umum Matchmaking}
% Photon memiliki fitur dan metode sendiri dalam mengelola sebuah \textit{game} multiplayer yang menggunakan \textit{framework}-nya.
% Photon Unity Networking merupakan class library yang dibungkus menjadi framework untuk mengelola pertukaran data sinkronisasi antar klien melalui Photon Cloud. 
% Arsitektur aplikasi permainan jak meuprang ini pada umumnya mempertemukan player pada lobby terlebih dahulu sebagai syarat memulai permainan dengan memanfaatkan event dan state yang ada pada Photon Unity Networking, komponen data pemain dan \textit{room} tersimpan dalam \textit{Custom Properties}. Berikut adalah arsitektur sistem dari aplikasi permainan pada Gambar \ref{fig:rumum-arsi}.
% \newpage
% \begin{figure}
%     \centering
%     \includegraphics[width=10cm]{rancangan-arsitektur.png}
%     \caption{Rancangan Arsitektur Umum Matchmaking}
%     \label{fig:rumum-arsi}
% \end{figure}

\subsection{Class Diagram Network}
\noindent

Rancangan kelas diagram ini merupakan bagian penting dari fitur \textit{Synchronous Multiplayer} dan akan diimplementasikan sesuai dengan rancangan diagram kelas seperti gambar \ref{fig:class-network}.

Pada kelas NetworkSpawner berfungsi untuk melakukan instansiasi objek karakter pemain dan menghidupkan kembali jika karakter pemain dalam keadaan \textit{state} mati.

Pada kelas NetworkMatchmaking, kelas ini akan menunggu event dari kelas NetworkManager. NetworkMatchmaking tidak dapat berjalan tanpa dikendalikan oleh NetworkManager.

Photon memiliki fitur penyimpanan \textit{state} sementara pada \textit{server}-nya (Photon Cloud), fitur tersebut berguna untuk menyimpan informasi \textit{room}, skor \textit{leaderboard} tiap pemain.
Fiture tersebut yaitu Photon Room Properties dan Photon Player Properties.
Perbedaanya adalah, penyimpanan dalam Photon Room Properties dapat diketahui nilainya oleh klien dalam satu \textit{room} yang sama, tetapi jika ingin mendapatkan nilai informasi dalam sebuah \textit{room} melalui Photon Room Properties, maka \textit{masterclient} diroom yang bersangkutan harus melakukan pengaturan pada \textit{method} CustomPropertiesForLobby yang terdapat pada photon.
Sedangkan Photon Player Properties hanya dapat diakses ketika dalam satu room yang sama.

NetworkPlayerProperty dan NetworkRoomProperty bertugas untuk menyimpan nilai \textit{unique} yang terdapat pada Photon Player Properties dan Photon Room Properties.

\begin{figure}[h]
    \centering
    \includegraphics[width=8cm]{class-diagram-network.png}
    \caption{Class Diagram Network}
    \label{fig:class-network}
\end{figure}

\begin{enumerate}
    \item Activity Class Diagram Network
    \begin{figure}[h]
        \centering
        \includegraphics[width=8cm]{activity-class-network.png}
        \caption{\textit{Activity Class Diagram Network}}
        \label{fig:activity-class-network}
    \end{figure}
    \newpage
    Dapat dilihat pada gambar \ref{fig:activity-class-network} digambarkan alur kerja atau aktivitas sebuah system \textit{photon unity networking} yang dimulai dengan terkoneksi internet kemudian sistem akan melakukan koneksi ke \textit{photon}. Jika gagal menghubungkan maka sistem akan mengembalikan pemain ke layar menu, jika berhasil player akan memasuki lobby.
    Pada saat dilobby pemain akan memasuki kondisi kedua, menghubungkan pemain ke dalam game server, jika sudah ter-\textit{connect} maka pemain akan memasuki \textit{room} yang sudah dihubungkan. Jika semua activity sudah dilakukan dan pemain sudah menghabiskan pada \textit{room} terssebut maka pemain akan dikembalikan ke main menu.
\end{enumerate}

\subsection{\textit{Class Diagram Game}}
\begin{figure}[h]
   \centering
   \includegraphics[width=10cm]{class-diagram-game.png}
    \caption{\textit{Class Diagram Game}}
    \label{fig:class-diagram-game}
\end{figure}

Pada gambar \ref{fig:class-diagram-game} perancangan sistem pembuatan game 3d terdiri dari, kontrol berfungsi untuk pegerakan arah dan posisi kamera karakter \textit{player}. Pada atribut karakter, karakter memiliki darah, pergerakan, bidik dan tembak  dan pada senjata, jika \textit{player} menembak \textit{player} lain maka player tersebut akan berkurang darahnya sebesar 20 begitu juga sebaliknya jika \textit{player} musuh menembak maka \textit{player} yang dimainkan akan berkurang darahnya sebesar 20.
\newpage
\begin{enumerate}
    \item \textit{Activity Class Diagram Game}
    \begin{figure}[h]
        \centering
        \includegraphics[width=10cm]{blok-diagram-game.png}
        \caption{\textit{Blok Diagram Game}}
        \label{fig:aclass-diagram-game}
    \end{figure}

    Dapat dilihat pada gambar \ref{fig:aclass-diagram-game} digambarkan alur kerja atau aktivitas sebuah sistem game yang dimulai \textit{start game} untuk memulai permainan kemudian menunggu \textit{loading} untuk memasuki permainan, sistem akan menginisialisasi objectnya terlebih dahulu seperti map, senjata dan point.
    Jika inisialisasi sudah selesai game akan dimulai dan pemain spawn pada saat game dimulai, jika pemain bertemu pemain lain dan bertempur akan ditemukan dua kondisi, jika mati pemain akan \textit{respawn} ulang dengan waktu 5 detik, jika pemain memenangkan pertempuran maka pemain mendapatkan point.
    Ketika game sudah habis waktu makan game sudah berakhir dan akan mamasukin akhir game permainan.
\end{enumerate}
\newpage

\section{Metode Penelitian}
\noindent

Photon memiliki fitur dan metode sendiri dalam mengelola sebuah \textit{game} multiplayer yang menggunakan \textit{framework}-nya. Metode penelitian yang menggunakan framework photon unity dapat digambarkan ke dalam bentuk flowchart seperti gambar .
        \begin{figure}[h]
         \centering
         \caption{flowchart Algoritma Matchmaking}
         \includegraphics[width=10cm]{flowchart-matchmaking}
         \label{fig:algoritmamatmaching}
         \end{figure}

Mengacu pada gambar \ref{fig:algoritmamatmaching}, algoritma membutuhkan daftar \textit{room} yang didapatkan dari \textit{interface} photon \textit{Behavior}. Jika belum ada \textit{room} yang tersedia pada daftar \textit{room}, maka dilakukan fungsi detil dari algoritma matchmaking yaitu SearchRoomMatchmaking(). 

% Berikut detil penjelasan flowchart pada gambar \ref{fig:detail-mm-1} dan gambar \ref{fig:detail-mm-2}.
% \begin{figure}[h]
%             \centering
%             \caption{Detail Algoritma Matchmaking 1}
%             \includegraphics[width=10cm]{detail-algoritma-matmaking1.png}
%             \label{fig:detail-mm-1}
%             \end{figure}
    
    
%             Mengacu pada Gambar \ref{fig:detail-mm-1}, ketika fungsi ini dipanggil, 
%     maka ia akan memulai perulangan sebanyak daftar room yang 
%     tersedia. Pencarian bertujuan untuk mencari room yang memiliki 
%     status sedang tidak bermain (waiting). Jika tidak menemukan 
%     room yang tersedia dan ber-status tidak sedang bermain (waiting) 
%     maka akan dilakukan pemanggilan fungsi 
%     RealocateRoomMatchmaking() yang dijelaskan lebih detil pada 
%     Gambar \ref{fig:detail-mm-2}. Jika sebaliknya maka pemain akan bergabung ke 
%     dalam room.
    
%     \newpage
%     \begin{figure}[h]
%         \centering
%         \caption{Detail Algoritma Matchmaking 2}
%         \includegraphics[width=10cm]{detail-algoritma-matmaking2.png}
%         \label{fig:detail-mm-2}
%         \end{figure}

        \section{Teknik Pengujian}
Teknik pengujian yang digunakan yaitu blackbox. Pengujian Black Box pada fungsional sistem yang terdapat pada aplikasi permainan jak meuprang.

    \begin{table}[h]
    \centering
    \begin{tabular}{|l|l|l|l|}
    \hline
    \multicolumn{1}{|c|}{NO} & \multicolumn{1}{c|}{Aktivitas Pengujian} & \multicolumn{1}{c|}{Hasil yang diharapkan} & \multicolumn{1}{c|}{Kesimpulan} \\ \hline
    1                        & Tombol Buat Room                         & Membuka panel "Room Panel"                 &                                 \\ \hline
    2                        & Tombol Join Room                         & Membuka panel "Lobby Panel"                &                                 \\ \hline
    3                        & Tombol Exit Room                         & Kembali ke menu utama                      &                                 \\ \hline
    4                        & Tombol About Game                        & Menampilkan popup about game               &                                 \\ \hline
    5                        & Tombol exit game                         & Keluar dari aplikasi game                  &                                 \\ \hline
    \end{tabular}
    \caption{Tabel Pengujian Black Box}
    \label{lab:tabel-pengujian}
    \end{table}
\newpage        
\section{Hasil yang diharapkan}
Hasil yang diharapkan pada penilitian ini antara lain :
\begin{enumerate}
    \item Keberhasilan dalam mengetahui kelayakan aplikasi game dengan metode \textit{blackbox testing}.
    \item Keberhasilan game terhubung dengan unity python networking yang dapat dimainkan secara multiplayer online.
    \item Laporan tugas akhir mahasiswa jurusan Teknologi Informasi Dan Komputer.
\end{enumerate}
\chapter*{JADWAL KEGIATAN PENELITIAN}
\addcontentsline{toc}{chapter}{JADWAL KEGIATAN PENELITIAN}
Hendaknya   dikemukakan   jenis-jenis   kegiatan   yang   direncanakan   dan    jadwalnya, mulai persiapan sampai penelitian berakhir pada penyusunan laporan. Jenis
kegiatan yang ditulis pada jadwal harus sesuai dengan langkah-langkah yang dilakukan pada metodologi.


\begin{table}[!ht]
\caption{Jadwal Kegiatan}
\label{tab:jadwal-kegiatan}
\begin{tabular}{|c|m{5cm}|p{.2cm}|p{.2cm}|p{.2cm}|p{.2cm}|p{.2cm}|p{.2cm}|p{.2cm}|p{.2cm}|p{.2cm}|p{.2cm}|p{.2cm}|p{.2cm}|}
\hline
\multirow{2}{*}{No}	& 
\multicolumn{1}{c|}{\multirow{2}{*}{Kegiatan}}	& 
\multicolumn{4}{c|}{Feb} & 
\multicolumn{4}{c|}{Mar} &
\multicolumn{4}{c|}{Apr} \\ \cline{3-14}
& & 
\multicolumn{1}{c|}{1} &
\multicolumn{1}{c|}{2} &
\multicolumn{1}{c|}{3} &
\multicolumn{1}{c|}{4} &
\multicolumn{1}{c|}{1} &
\multicolumn{1}{c|}{2} &
\multicolumn{1}{c|}{3} &
\multicolumn{1}{c|}{4} &
\multicolumn{1}{c|}{1} &
\multicolumn{1}{c|}{2} &
\multicolumn{1}{c|}{3} &
\multicolumn{1}{c|}{4} \\ \hline
1  & Pengumpulan Data 			& \cellcolor{gray!75} & \cellcolor{gray!75} & & & & & & & & & & \\ \hline
2  & Identifikasi Masalah 		& & & \cellcolor{gray!75} & & & & & & & & & \\ \hline
3  & Analisis Kebutuhan Sistem 	& & \cellcolor{gray!75} & \cellcolor{gray!75} & \cellcolor{gray!75} & & & & & & & & \\ \hline
4  & Membuat Rancangan Sistem 	& & & & \cellcolor{gray!75} & \cellcolor{gray!75} & & & & & & & \\ \hline
5  & Rancang Bangun Program 	& & & & & & \cellcolor{gray!75} & \cellcolor{gray!75} & \cellcolor{gray!75} & & & & \\ \hline
6  & Uji coba program (\textit{testing}) 			& & & & & & & & \cellcolor{gray!75} & \cellcolor{gray!75} & & & \\ \hline
7  & Revisi Konsep, Desain Rancangan, Code Program & & & & & & & & & \cellcolor{gray!75} & \cellcolor{gray!75} & & \\ \hline
8  & Implementasi Program 					& & & & & & & & & & & \cellcolor{gray!75} & \\ \hline
9  & Pembimbingan Penulisan Naskah Skripsi & & & & & \cellcolor{gray!75} & \cellcolor{gray!75} & \cellcolor{gray!75} & \cellcolor{gray!75} & \cellcolor{gray!75} & & & \\ \hline
10 & Penulisan Akhir Laporan 				& & & & & & & & & & \cellcolor{gray!75} & & \\ \hline
11 & Pendadaran 							& & & & & & & & & & & & \cellcolor{gray!75} \\ \hline
\end{tabular}
\end{table}



Catatan : 
\begin{itemize}
\item Kolom yang menyebutkan bulan, tuliskan berapa bulan dapat diselesaikan dan serta dengan minggunya.
\item Nama kegiatan dan Jumlah kegiatan disesuaikan dengan rencana kegiatan yang akan dilakukan
\end{itemize}
\chapter*{RENCANA ANGGARAN PENELITIAN}
\addcontentsline{toc}{chapter}{RENCANA ANGGARAN PENELITIAN}
Rencana anggaran penelitian pada pembuatan penelitian “Rancang Bangun Game Mulitplayer Online First Person Shooter(FPS) 3D Menggunakan Photon Unity Networking”.

\begin{table}[h]
    \centering
    \begin{tabular}{|llcl|l|ll}
    \cline{1-5}
    \multicolumn{1}{|c|}{NO} & \multicolumn{1}{c|}{Uraian}          & \multicolumn{1}{c|}{Volume}  & \multicolumn{1}{c|}{Harga(Rp)} & Jumlah(Rp)    &  &  \\ \cline{1-5}
    \multicolumn{1}{|l|}{1}  & \multicolumn{1}{l|}{Kuota}           & \multicolumn{1}{c|}{6 Bulan} & Rp.30.000                      & Rp. 180.000   &  &  \\ \cline{1-5}

    \multicolumn{4}{|c|}{Jumlah}                                                                                                    & Rp. 180.000 &  &  \\ \cline{1-5}
    \end{tabular}
    \caption{Anggaran Penelitian}
    \label{lab:label-anggaran}
    \end{table}

\begin{thebibliography}{9}
\bibitem{Sarwodi}
Sarwodi, S.S., Wardhono, W.S. and Akbar, M.A., 2020. \emph{Penerapan Multiplayer Pada Gim Tower Defense Menggunakan Photon Unity.}, Jurnal Pengembangan Teknologi Informasi dan Ilmu Komputer e-ISSN, 2548, p.964X.

\bibitem{Ansori}
Pratama, R.N., Ansori, A.S.R. and Dinimaharawati, A., 2021 \emph{Pembuatan Multiplayer Game Ucing Beling Menggunakan Asset Store Mirror}, eProceedings of Engineering, 8(5).
Wesley, Massachusetts, 2nd ed.

\bibitem{androidsdk}
Download Android Studio and SDK Tools | Android 
Developers," Google, [Online]. Available: 
http://developer.android.com/sdk/index.html. [Accessed 08 
Maret 2023].

\bibitem{pun}
Photon Unity Networking 2," Unity, 
[Online]. Available : https://doc-api.photonengine.com/en/pun/v2/. [Accessed 08 Maret 2023]

\bibitem{fps}
Ramadhan, I. and Purwanto, A., 2020. \emph{Pengembangan Teknologi Game Indonesia Untuk Permainan First Person Shooter (Fps) 3d Multiplayer “Code To Shoot” Menggunakan Unity Network (Unet) Berbasis Mobile.} Jurnal Teknologi Informasi Universitas Lambung Mangkurat (JTIULM), 5(2), pp.39-48.

\bibitem{fps2013}
Halim, M., 2013. \emph{PEMBUATAN GAME “THE LAST MISSION” DENGAN MENGGUNAKAN FPS CREATOR} (Doctoral dissertation, Universitas AMIKOM Yogyakarta).

\bibitem{asyncyuhu}
Muhammad, A., 2015. \emph{Rancang Bangun Synchronous PVP Multiplayer Online Dalam Game Sosial Rangers Companion Dengan Menggunakan Unity Dan Framework Photon Unity Networking Pada Perangkat Android}.(Doctoral dissertation, Institut Technology Sepuluh Nopember).
\end{thebibliography}
\addcontentsline{toc}{chapter}{DAFTAR PUSTAKA}

\end{document}
