\chapter{PENDAHULUAN}
\section{Latar Belakang Masalah}
 
Game atau permainan adalah aktivitas yang dilakukan untuk tujuan hiburan atau kompetisi, dengan aturan yang telah ditentukan dan biasanya memiliki elemen interaktif yang melibatkan satu atau lebih peserta. Game sering kali melibatkan strategi, kecepatan, keterampilan, atau ketangkasan fisik, tergantung pada jenisnya. Tujuan dari game adalah untuk mencapai kemenangan, skor tinggi, atau hanya untuk kesenangan semata. Game bisa dimainkan secara individu atau dalam kelompok, dan dapat berupa permainan fisik seperti sepak bola, permainan papan seperti catur atau permainan video seperti Mario Bros.

First Person Shooter merupakan sebuah permainan peperangan menggunakan senjata api dengan sudut pandang orang pertama dan hanya menampilkan senjata yang dipegang.
Dalam permainan FPS, pemain biasanya melawan musuh secara langsung dalam pertempuran yang cepat dan intens. Senjata api menjadi alat utama pemain dalam memerangi musuh.
Agar game First Person Shooter (FPS) lebih menarik dimainkan, peniliti menambahkan fitur multiplayer agar dapat dimainkan bersama sama secara online yang dapat terhubung dimana saja dengan menggunakan koneksi internet. Untuk membuat fitur multiplayer peniliti menggunakan game engine unity dan framework photo unity networking.

Sistem multiplayer pada sebuah \textit{game} membuat \textit{game} tersebut menjadi lebih interaktif dan menarik untuk dimainkan. Dalam sebuah gim jika pemain memilih untuk single player maka pemain tersebut akan berhadapan dengan lawan NPC (Non Playable Character) sedangkan jika multiplayer maka pemain tersebut akan berhadapan dengan pemain lain.


Penelitian ini mengusulkan sebuah \textit{game} dengan memanfaatkan koneksi via internet yang dapat memainkan \textit{game} bertema first person shooter, dimana pemain bersaing secara real (nyata) dan lebih menantang di mana minimal ada 2 pemain yang akan bertemu dalam satu room.

Berdasarkan penjabaran diatas, maka diusulkan sebuah judul skripsi yang mengimplementasikan koneksi internet menggunakan photon unity asset pada \textit{game} first person shooter 3D yang dapat dimainkan menggunakan perangkat android dengan judul "Implementasi \textit{game} First Person Shooter (FPS) 3D Multiplayer "Jak Meuprang" menggunakan photon unity network (PUN)".
\textit{game} ini akan dibuat multiplayer menggunakan fitur dari unity \textit{game} engine yaitu photon unity networking, pemain tidak perlu lagi melakukan set alamat ip untuk bisa bermain secara multiplayer.

\section{Rumusan Masalah}
Berdasarkan latar belakang masalah yang telah diuraikan, maka didapat perumusan masalah sebagai berikut :
\begin{enumerate}
	\item Bagaimana mengimplementasikan photon unity networking pada \textit{game} first person shooter (fps) ?
	\item Bagaimana pemain dapat terhubung ke dalam mode multiplayer?
	\end{enumerate}	
\section{Tujuan Penelitian}
Tujuan dari pembuatan skripsi ini adalah untuk membuat \textit{game} android yang mengimplementasikan photon unity networking(pun) pada \textit{game} first person shooter untuk dapat bermain secara multiplayer dengan menggunakan koneksi internet dan dapat dimainkan dimana saja.

\section{Batasan Masalah}
Pada penelitian ini terdapat batasan masalah dengan maksud untuk mempermudah penulis, adapun batasan masalah pada penelitian ini sebagai berikut:
\begin{enumerate}
	\item Pembuatan \textit{game} ini akan menggunakan IDE Unity dan bahasa pemrograman Csharp.
	\item Total maksimum CCU (Concurent Users) yang dapat terhubung ke Photon Cloud yaitu 20 CCU.
	\item Hanya dapat dimainkan diplatform pc.
\end{enumerate}

\section{Manfaat Penelitian}
Manfaat dari penilitian ini antara lain adalah : 
\begin{enumerate}
	\item Sebagai sarana hiburan untuk para pengguna.
	\item Sebagai bentuk implementasi konsep photon unity networking pada \textit{game} fps jak meuprang.
\end{enumerate}