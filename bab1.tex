\chapter{PENDAHULUAN}
\section{Latar Belakang Masalah}
 
Game mobile adalah game yang dirancang dan dimainkan di perangkat mobile seperti smartphone atau tablet. game mobile dapat bervariasi dari game sederhana seperti game puzzle atau game arcade, hingga game yang lebih kompleks seperti game rpg atau game open world. Banyak game mobile yang dirancang untuk dimainkan dalam sesi singkat, sehingga cocok dimainkan saat sedang menunggu sesuatu. Seiring dengan perkembangan teknologi, game mobile juga semakin berkembang dalam hal kualitas grafis, fitur, dan gameplay yang semakin canggih. Beberapa game mobile bahkan memiliki kualitas yang setara dengan game PC atau konsol, sehingga semakin diminati oleh para pengguna mobile.

First Person Shooter merupakan sebuah permainan peperangan menggunakan senjata api dengan sudut pandang orang pertama dan hanya menampilkan senjata yang dipegang.
Dalam permainan FPS, pemain biasanya melawan musuh secara langsung dalam pertempuran yang cepat dan intens. Senjata api menjadi alat utama pemain dalam memerangi musuh.
Hal ini peneliti paparkan berdasarkan observasi yang peneliti lakukan dengan menginstal beberapa 
game FPS serta memainkannya secara langsung seperti game PUBG Mobile, CODM, Apex Mobile, Fortnite dan sebagainya.

Sistem multiplayer pada sebuah game membuat game tersebut menjadi lebih interaktif dan menarik untuk dimainkan. Dalam sebuah gim jika pemain memilih untuk single player maka pemain tersebut akan berhadapan dengan lawan NPC (Non Playable Character) sedangkan jika multiplayer maka pemain tersebut akan berhadapan dengan pemain lain.
Game multiplayer terbagi menjadi multiplayer offline dan multiplayer online. Pada game multiplayer offline pemain dapat berinteraksi dengan pemain lain tanpa harus terkoneksi ke internet
sedangkan multiplayer online player dapat memainkan game dimanapun dengan jarak yang jauh sekalipun.

Penelitian ini mengusulkan sebuah game dengan memanfaatkan koneksi via internet yang dapat memainkan game bertema first person shooter, dimana pemain bersaing secara real (nyata) dan lebih menantang di mana minimal ada 2 pemain yang akan bertemu dalam satu room.

Berdasarkan penjabaran diatas, maka diusulkan sebuah judul skripsi yang mengimplementasikan koneksi internet menggunakan photon unity asset pada game first person shooter 3D yang dapat dimainkan menggunakan perangkat android dengan judul "Implementasi Game First Person Shooter (FPS) 3D Multiplayer "Jak Meuprang" menggunakan photon unity network (PUN)".
Game ini akan dibuat multiplayer menggunakan fitur dari unity game engine yaitu photon unity networking, pemain tidak perlu lagi melakukan set alamat ip untuk bisa bermain secara multiplayer.

\section{Rumusan Masalah}
Berdasarkan latar belakang masalah yang telah diuraikan, maka didapat perumusan masalah sebagai berikut :
\begin{enumerate}
	\item Bagaimana mengimplementasikan photon unity networking pada game first person shooter (fps) ?
	\item Bagaimana pemain dapat terhubung ke dalam mode multiplayer?
	\end{enumerate}	
\section{Tujuan Penelitian}
Tujuan dari pembuatan skripsi ini adalah untuk membuat game android yang mengimplementasikan photon unity networking(pun) pada game first person shooter untuk dapat bermain secara multiplayer dengan menggunakan koneksi internet dan dapat dimainkan dimana saja.

\section{Batasan Masalah}
Pada penelitian ini terdapat batasan masalah dengan maksud untuk mempermudah penulis, adapun batasan masalah pada penelitian ini sebagai berikut:
\begin{enumerate}
	\item Pembuatan game ini akan menggunakan IDE Unity dan bahasa pemrograman Csharp.
	\item Total maksimum CCU (Concurent Users) yang dapat terhubung ke Photon Cloud yaitu 20 CCU.
	\item Hanya dapat dimainkan diplatform android.
\end{enumerate}

\section{Manfaat Penelitian}
Manfaat dari penilitian ini antara lain adalah : 
\begin{enumerate}
	\item Sebagai sarana hiburan untuk para pengguna.
	\item Sebagai bentuk implementasi konsep photon unity networking pada game fps jak meuprang.
\end{enumerate}