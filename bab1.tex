\chapter{PENDAHULUAN}
\section{Latar Belakang Masalah}
Latar belakang masalah dalam suatu penelitian sangat penting karena di bagian ini diperkenalkan pokok masalah yang akan diteliti. Dalam latar belakang  masalah juga harus diuraikan alasan secara ilmiah topik  tersebut dipilih. Dalam hal ini permasalahan harus jelas terungkap melalui argumentasi dan fakta tentang  perlunya topik tersebut  diteliti. Penyusunan latar belakang masalah  dapat dilakukan melalui dua pendekatan yaitu:
\begin{enumerate}
\item Diawali dengan pemikiran teoritis kemudian mengarah ke fakta emperik;
\item Diawali dengan dunia emperik kemudian mengarah ke pemikiran teoritis.
\end{enumerate}    
Dalam latar belakang harus ada idealitas, ada problem (gap), dan ada gagasan untuk menyelesaikan   (urgensi penelitian) tentnag hasil. Isi latar pelakang masalah boleh juga ditampilkan kutipan dari artikel lainnya yang berkaitan erat dengan penelitian yang akan dilakukan. Hal ini menjadi pendukung terhadap penelitian yang akan dilakukakan. Kutipan yang dimaksud bukan definisi, tetapi hasil penelitian pihak lain yang berkaitan dengan penelitian yang akan dilakukan.
	Pola umum  yang biasa digunakan  untuk penyususnan latar belakang masalah  adalah sebagai berikut:
\begin{enumerate}
\item Awali dengan general statement \\
Latar belakang diawali dengan statemen  umum (biasanya berupa fakta yang sudah diketahui masyarakat luas) yang terkait dengan pokok masalah yang akan diteliti.
\item Kemukakan pokok masalah \\
Dalam hal ini dikemukakan masalah atau landasan awal yang dapat menghubungkan statemen awal dengan penelitian yang akan dilakukan yang disertai data-data pendukung yang menunjukkan masalah data dapat diperoleh atau berupa hasil wawancara dan/atau hasil observasi.
\item Bahas masalah secara lebih spesifik \\
Masalah dibahas secara lebih mengerucut dan spesifik agar pembaca dapat memahami arah penelitian secara jelas yang dituangkan dalam karya tulis. 
\item Relevansikan dengan tujuan penelitian \\
Pada bagian ini yang merupakan bagian akhir latar belakang, berisi resolusi terhadap masalah yang telah dibahas pada pargraf sebelumnya.
\end{enumerate}	

\section{Rumusan Masalah}
Rumusan masalah merupakan pernyataan tentang masalah yang jawabannya/ pemecahannya dapat dicari melalui penelitian. Kejelasan latar belakang timbulnya masalah akan memudahkan perumusan masalah dan menentukan batasannya. Permasalahan hendaknya  ditulis  dalam  bentuk  deklaratif,  tegas, jelas, singkat, dan padat. Rumusan masalah dapat dinyatakan dalam bentuk kalimat pernyataan  dan dapat pula dinyatakan dalam bentuk pertanyaan. Umumnya, masalah dalam penelitian sering dinyatakan dalam bentuk kalimat tanya. Kalimat tanya yang dimaksud harus berupa pertanyaan penelitian yaitu pertanyaan yang jawabannya membutuhkan data, bukan pertanyaan biasa,  yaitu pertanyaan  dapat dijawab tanpa harus ada data.  
Pertanyaan dalam rumusan masalah harus dapat diprediksi jawabannya dan harus dipastikan adanya data untuk menjawab pertanyaan tersebut. Jawaban terhadap masalah penelitian berupa hasil, buka cara, tahapan, dan bukan pula proses. Jawaban terhadap rumusan masalah yang dikemukakan pada bab satu  dipastikan  dapat ditemukan  jawabannya  dalam bab hasil dan pembahasan dan bab simpulan hasil penelitian. 

\section{Tujuan Penelitian}
Pada bagian ini diuraikan   tujuan   secara   logis, sistematis, dan   harus   berkaitan   langsung   dengan permasalahan penelitian serta memuat tentang rincian variabel yang akan diteliti atau diukur untuk mencapai luaran yang diharapkan.Tujuan merupakan “Janji peneliti “dalam melaksanakan suatu kegiatan spesifik yang bersifat operasional (dapat ditulis menggunakan kata kerja) secara jelas yang  didukung dengan data atau penalaran yang mantap. Pada tujuan penelitian  diuraikan permasalahan  yang akan diangkat dalam penelitian dan   disebutkan   luaran   yang   akan   dihasilkan   sebagai   solusi   untuk   pemecahkan masalah yang telah dirumuskan.

\section{Batasan Masalah}
Batasan masalah atau sering disebut dengan ruang  lingkup penelitian      harus  menjelaskan luasnya area penelitian yang akan dieksplorasi dalam penelitian tersebut dan menentukan parameter dalam penelitian yang akan dilaksanakan. Pada bagian ini   dibatasi  permasalahan pada pokok persoalan  yang akan dipecahkan  sesuai dengan masalah yang telah dirumuskan dan  solusi yang ditawarkan pada latar belakang masalah yang telah diuraikan sebelumnya. Batasan  masalah  harus  menjelaskan  tentang  dapat  tidaknya  perumusan masalah tersebut dicapai.

\section{Manfaat Penelitian}
Manfaat penelitian merupakan konstribusi hasil penelitian terhadap pihak lain, terutama ditujukan bagi pengembangan ilmu atau pelaksanaan pembangunan dalam arti luas. Uraian bagian ini berisi alasan  bahwa penelitian terhadap masalah yang dipilih memang memberikan manfaat kepada pihak tertentu. Manfaat penelitian  menyatakan secara eksplisit berbagai pihak yang dapat memanfaatkan hasil penelitian yang dimaksud.