\chapter{TINJAUAN PUSTAKA}
\section{\textit{State of the Art}}
\textit{State of the Art} merupakan langkah untuk menunjukkan keterkinian penelitian yang dilakukan dan hubungannya dengan penelitian sebelumnya. Pada hakikatnya, hasil penelitian seorang peneliti bukanlah satu penemuan baru yang   berdiri   sendiri,   melainkan   sesuatu   yang   berkaitan   dengan   hasil   penelitian sebelumnya. Pada bagian ini harus dielaborasikan hasil peneliti terdahulu yang berkaitan dengan masalah yang akan diteliti sehingga memberikan gambaran  perkembangan  pengetahuan.   Dalam \textit{state of the art} harus diuraikan tentang penelitian-penelitian yang  diperoleh dari artikel  yang telah dipublikasikan melului jurnal-jurnal yang ber-ISSN  dan mempunyai  hubungan yang erat   dengan penelitian yang akan dilakukan.  Sumber yang digunakan di bagian ini berupa artikel hasil penelitian yang dijadikan acuan dan tidak digunakan sumber  berupa buku.  Di bagian ini   dituliskan juga rencana penelitian yang akan dilakukan  dan dijelaskan perbedaan pada (variabel, algoritma atau metode) dari  penelitian yang akan dilakukan dengan penelitian sebelumnya serta dijelaskan solusi  yang akan digunakan dalam menyelesaikan penelitian yang dimaksud.  Pemaparannya ditampilkan  dalam bentuk tabel.  Contoh pemaparan \textit{state of the art} dapat dilihat pada Tabel \ref{tb:stateoftheart} berikut.

\begin{landscape}
\begin{table}[h]
\caption{Paparan \textit{State of the Art}}
\label{tb:stateoftheart}
\begin{tabular}{| c | c | L{4cm} | L{2.5cm} | L{4cm} | L{3cm} | L{3cm} |}
\hline
No &
Penulis/Tahun &
\multicolumn{1}{c|}{Judul Artikel} &
\multicolumn{1}{c|}{Metode yang digunakan} &
\multicolumn{1}{c|}{Hasil yang diperoleh} &
\multicolumn{1}{c|}{Persamaan} &
\multicolumn{1}{c|}{Perbedaan} \\ \hline
1 	&
 	&
 	&
 	&
 	&
 	&
 	\\ \hline
2 	&
 	&
 	&
 	&
 	&
 	&
 	\\ \hline
3 	&
 	&
 	&
 	&
 	&
 	&
 	\\ \hline
4 	&
 	&
 	&
 	&
 	&
 	&
 	\\ \hline
5 	&
 	&
 	&
 	&
 	&
 	&
 	\\ \hline
6 	&
 	&
 	&
 	&
 	&
 	&
 	\\ \hline
7 	&
 	&
 	&
 	&
 	&
 	&
 	\\ \hline
8 	&
 	&
 	&
 	&
 	&
 	&
 	\\ \hline
9 	&
 	&
 	&
 	&
 	&
 	&
 	\\ \hline
10 	&
 	&
 	&
 	&
 	&
 	&
 	\\ \hline 		 	 	
\end{tabular}
\end{table}
\end{landscape}

\section{Tinjauan Teoritis}
Bagian ini  berisi uraian tentang alur pemikiran dan perkembangan keilmuan topik kajian. Tinjauan  pustaka  hendaklah  disusun  sesuai dengan urutan perkembangan cabang ilmu pengetahuan yang terkandung di dalamnya. 

Tinjauan teoritis berisi pula ulasan tentang simpulan  hasil penelitian  dari artikel yang diajdikan referensi dalam penlitian yang akan dilakukan.   Bagian ini juga diuraikan  teori yang benar-benar menunjang penelitian yang akan dilaksanakan, mengacu pada ruang lingkup yang telah diuraikan sebelumnya dan ditulis secara sistematis sehingga jelas keterkaitannya dengan penelitian yang akan dilakukan. Teori penunjang menguraikan dasar-dasar teori, temuan, dan bahan  dari pustaka ilmiah lain, yang dijadikan landasan untuk melakukan proyek akhir yang diusulkan. Uraian dalam Teori Penunjang menjadi landasan untuk menyusun kerangka atau konsep yang akan digunakan dalam proyek akhir. Pada pembahasan teori penunjang, jangan lupa untuk menyebutkan semua kutipan dengan rujukan yang jelas seperti ini

Sumber yang  digunakan sebagai acuan  pada bagian ini tidak hanya  artikel yang duplikasikan melalui jurnal yang ber-ISSN, tetapi juga harus ada sumber  dan dari buku-buku yang ber-ISSN. Tahun paling rendah dari sumber yang dijadika acuan 10 tahun ke bawah. Artinya, Kalau Anda menulis di tahun 2021, tahun terendah sumber yang Anda gunakan tahun 2011 ke atas dan tidak boleh digunakan sumber di bawah tahun tersebut. Jumlah sumber  yang digunakan pada bagian ini minimal 5 sumber. Semua sumber yang dijadikan acuan harus dimasukkan dalam daftat pustaka.