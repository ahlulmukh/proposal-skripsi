\chapter{TINJAUAN PUSTAKA}
\section{\textit{State of the Art}}
\textit{State of the Art} Dalam penyusunan penilitian ini, peniliti mengambil beberapa referensi terdahulu sebagai panduan penulis untuk penilitian yang dilakukan, yang kemudian  akan menjadi acuan dan perbedaan dari penilitian yang akan dilakukan dengan penilitian sebelumnya. Pemaparan \textit{State of the Art} dapat dilihat pada tabel \ref{tb:stateoftheart} berikut.

\begin{landscape}
\begin{table}[ht!]
\caption{Paparan \textit{State of the Art}}
\label{tb:stateoftheart}
\begin{tabular}{| c | c | L{4cm} | L{4cm} | L{4cm} | L{3cm} | L{3cm} |}
\hline
No &
Penulis/Tahun &
\multicolumn{1}{c|}{Judul Artikel} &
\multicolumn{1}{c|}{Metode yang digunakan} &
\multicolumn{1}{c|}{Hasil yang diperoleh} &
\multicolumn{1}{c|}{Persamaan} &
\multicolumn{1}{c|}{Perbedaan} \\ \hline
1 	& \makecell{IKROM AULIA FAHDI\\ (2016)}
 	&ASYNCHRONOUS MULTIPLAYER CARD GAME PADA GAME MONSTER KING
 	& blacbox
 	&Mekanisme asynchronous berhasil diimplementasikan menggunakan Photon Realtime.
 	&Persamaannya yaitu menggunakan unity photon
 	&Perbedaannya terdapat pada frameworky yang digunakan peniliti terdahulu menggunakan photon realtime
 	\\ \hline
2 	& \makecell{Ibnu Ramadhan\\, Agung Purwanto\\ dan Nurahman\\2020}
 	& PENGEMBANGAN TEKNOLOGI GAME INDONESIA UNTUK PERMAINAN FIRST PERSON SHOOTER (FPS) 3D MULTIPLAYER “CODE TO SHOOT” MENGGUNAKAN UNITY NETWORK (UNET) BERBASIS MOBILE
 	& blacbox
 	& Game ini dapat dimainkan secara multiplayer tanpa perlu memasukan alamat IP karena fitur uNet dapat bekerja dengan baik. Selain itu game ini juga sudah dapat dimainkan menggunakan platform mobile (Android)
 	& Sama sama game bergenre first person shooter(fps)
 	& Perbedaannya peniliti menggunakan unet sebagai multiplayer platform
 	\\ \hline		
\end{tabular}
\end{table}
\end{landscape}

\begin{landscape}
	\begin{table}[ht!]
	\ContinuedFloat
	\caption{Paparan \textit{State of the Art}}
	\begin{tabular}{| c | c | L{3.2cm} | L{4cm} | L{4cm} | L{3cm} | L{3cm} |}
	\hline
	No &
Penulis/Tahun &
\multicolumn{1}{c|}{Judul Artikel} &
\multicolumn{1}{c|}{Metode yang digunakan} &
\multicolumn{1}{c|}{Hasil yang diperoleh} &
\multicolumn{1}{c|}{Persamaan} &
\multicolumn{1}{c|}{Perbedaan} \\ \hline
	3 	& \makecell{Shena Star Sarwodi\\ Wibisono Sukmo Wardhono \\ Muhammad Aminul \\Akbar (2020)}
		 & Penerapan Multiplayer Pada Gim Tower Defense Menggunakan Photon Unity
		 & whitebox, fps, delay dan Game Experience Questioner (GEQ).
		 & Dengan menerapkan Photon Unity Networking pada gim tower defense maka dapat diimplementasikan sebuah fitur yang dapat meningkatkan interaktivitas dan ketertarikan pemain pada gim yaitu fitur multiplayer
		 &Persamaannya yaitu menggunakan unity photon.
		 &Perbedaannya terdapat pada metode pengujian.
		 \\ \hline
	4 	& \makecell{Ryan Nanda Pratama\\,  Anton Siswo Raharjo Ansori\\ Ashri Dinimaharawati\\(2021)}
		 & PEMBUATAN MULTIPLAYER GAME UCING BELING MENGGUNAKAN ASSET STORE MIRROR
		 & blacbox
		 & Pada multiplayer game Ucing Beling dapat dimainkan secara realtime dan berjalan dengan sesuai yang diharapkan
		 & Persamaannya sama sama multiplayer.
		 & Perbedaannya peniliti menggunakan asset store mirror sebagai multiplayer platform
		 \\ \hline		
	\end{tabular}
	\end{table}
	\end{landscape}

\section{Tinjauan Teoritis}
\subsection{Unity}
Unity merupakan salah satu game engine paling populer saat ini. Penggunaan Unity dapat digunakan untuk mengembangkan konten interaktif seperti video game, 
visualisasi arsitektur, dan real-time 3D animasi. Unity menggunakan bahasa pemograman JavaScript dan 
Csharp \cite{Ansori}. Unity juga merupakan perangkat lunak yang digunakan untuk mengembangkan game multiplatform yang didesain secara user friendly 
(Iman, 2017). Keunggulan Unity adalah Unity 
dapat dengan mudah mengontrol objek-objek 
dalam gim atau aplikasi. Unity terdapat 2 jenis 
lisensi yaitu personal edition yang dapat diakses 
secara gratis dan professional edition yang 
diharuskan untuk membayar perbulan untuk 
mengaksesnya dengan beberapa fitur tambahan 
yang tidak terdapat di personal edition \cite{Sarwodi}. 

\subsection{Multiplayer}
Multiplayer merupakan fitur pada game dimana pemain bermain dengan lebih dari 1 orang yang bermain 
di lingkungan game yang sama dan waktu yang bersamaan. Game Multiplayer biasanya memberikan pilihan pada 
pemain untuk berbagi sumber daya sistem game atau menggunakan internet untuk bermain bersama dalam jarak 
jauh. Game Multiplaye yang terhubung dengan internet melibatkan pemain yang saling terhubung melalui server. 
Sedangkan Game Multiplayer dengan koneksi lokal yaitu, pemain saling terhubung secara langsung dengan 
pemain lainnya, pemain terkoneksi menggunakan jaringan peer to peer. Pada Game Multiplayer online memiliki 
beberapa jenis kategori diantaranya adalah Massively Multiplayer Online game (MMO), Massively Multiplayer 
Online First-person Shooter Games (MMOFPS), Massively Multiplayer online Real-time Strategy Games
(MMORTS), Massively Multiplayer Online Role-playing Games (MMORPG), Multiplayer Online Battle Arena
(MOBA).\cite{Ansori}

\subsection{Android SDK}
Android-SDK merupakan tools bagi para programmer yang 
ingin mengembangkan aplikasi berbasis android \cite{androidsdk}. Android SDK 
mencakup seperangkat alat pengembangan yang komprehensif. 
Android SDK terdiri dari debugger, libraries, handset
emulator,dokumentasi, contoh kode, dan tutorial. Persyaratan 
mencakup JDK, Apache Ant dan Python 2.2 atau yang lebih baru. 
IDE yang didukung secara resmi adalah Android Studio. Dengan 
menggunakan android studio ini pengembang dapat menggunakan 
teks editor untuk mengedit file Java dan XML serta menggunakan 
peralatan command line untuk Melakukan debug aplikasi Android, dll.


\subsection{Photon Unity Networking (PUN)}
Photon adalah sebuah framework pengembangan game multiplayer real-time yang cepat, ringan, dan fleksibel. Photon terdiri dari server dan beberapa SDK klien untuk platform utama.
Photon Unity Network (PUN) adalah solusi khusus Unity yang dihadirkan dengan tingkat yang lebih tinggi: matchmaking, panggilan balik yang mudah digunakan, komponen untuk sinkronisasi GameObjects, Remote Procedure Calls (RPCs), dan fitur serupa yang memberikan awal yang baik. Di luar itu, terdapat API yang solid dan luas untuk kontrol yang lebih canggih \cite{pun}.

\subsection{First Person Shooter(FPS)}
First Person Shooter (FPS) adalah salah satu jenis game yang saat ini sangat digemari terutama kalangan gamers muda. FPS merupakan game yang menggunakan sudut pandang orang pertama dimana pemain akan dibuat seolah-olah menjadi karakter utama dalam game dengan tampilan yang berpusat pada permainan disekitar senjata atau alat yang sedang digunakan \cite{fps}.

First person shooter merupakan jenis 3D game shooter yang menampilkan sudut pandang orang pertama dengan 
pemain yang melihat aksi melalui mata karakter permain. Tidak seperti orang ketiga yang terlihat dari bagian 
belakang atau samping, yang memungkinkan gamer untuk melihat karakter secara keseluruhan\cite{fps}.

FPS dikembangkan pada tahun 1973 melalui permainan ruang yang belum sempurna yaitu flight simulator, yang 
menampilkan sudut pandang orang pertama dengan mengarah lebih rinci ke simulator pesawat tempur, dikembangkan untuk pasukan AS pada akhir tahun 1970-an. Permainan ini tidak lagi tersedia untuk konsumen \cite{fps}.

\subsection{C\#}
C\# (C-sharp) adalah salah satu bahasa pemograman yang menggunakan Framework .NET. Sama seperti 
bahasa lainnya, C\# memiliki aturan pada syntax dan kode-kode yang bisa digunakan dalam pembuatan aplikasi. 
C\# cocok untuk dipelajari untuk pemula karena aturan syntax-nya lebih sederhana dibandingkan bahasa 
pemograman lainnya \cite{Ansori}.


