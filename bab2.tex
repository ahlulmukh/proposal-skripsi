\chapter{TINJAUAN PUSTAKA}
\section{\textit{State of the Art}}
\textit{State of the Art} Dalam penyusunan penilitian ini, peniliti mengambil beberapa referensi terdahulu sebagai panduan penulis untuk penilitian yang dilakukan, yang kemudian  akan menjadi acuan dan perbedaan dari penilitian yang akan dilakukan dengan penilitian sebelumnya. Pemaparan \textit{State of the Art} dapat dilihat pada tabel \ref{tb:stateoftheart} berikut.

\begin{landscape}
\begin{table}[ht!]
\caption{Paparan \textit{State of the Art}}
\label{tb:stateoftheart}
\begin{tabular}{| c | c | L{4cm} | L{4cm} | L{4cm} | L{3cm} | L{3cm} |}
\hline
No &
Penulis/Tahun &
\multicolumn{1}{c|}{Judul Artikel} &
\multicolumn{1}{c|}{Metode yang digunakan} &
\multicolumn{1}{c|}{Hasil yang diperoleh} &
\multicolumn{1}{c|}{Persamaan} &
\multicolumn{1}{c|}{Perbedaan} \\ \hline
1 	& \makecell{IKROM AULIA FAHDI\\ (2016)}
 	&ASYNCHRONOUS MULTIPLAYER CARD GAME PADA GAME MONSTER KING
 	& blacbox
 	&Mekanisme asynchronous berhasil diimplementasikan menggunakan Photon Realtime.
 	&Persamaannya yaitu menggunakan unity photon
 	&Perbedaannya terdapat pada frameworky yang digunakan peniliti terdahulu menggunakan photon realtime
 	\\ \hline
2 	& \makecell{Ibnu Ramadhan\\, Agung Purwanto\\ dan Nurahman\\2020}
 	& PENGEMBANGAN TEKNOLOGI GAME INDONESIA UNTUK PERMAINAN FIRST PERSON SHOOTER (FPS) 3D MULTIPLAYER “CODE TO SHOOT” MENGGUNAKAN UNITY NETWORK (UNET) BERBASIS MOBILE
 	& blacbox
 	& Game ini dapat dimainkan secara multiplayer tanpa perlu memasukan alamat IP karena fitur uNet dapat bekerja dengan baik. Selain itu game ini juga sudah dapat dimainkan menggunakan platform mobile (Android)
 	& Sama sama game bergenre first person shooter(fps)
 	& Perbedaannya peniliti menggunakan unet sebagai multiplayer platform
 	\\ \hline		
\end{tabular}
\end{table}
\end{landscape}

\begin{landscape}
	\begin{table}[ht!]
	\ContinuedFloat
	\caption{Paparan \textit{State of the Art}}
	\begin{tabular}{| c | c | L{3.2cm} | L{4cm} | L{4cm} | L{3cm} | L{3cm} |}
	\hline
	No &
Penulis/Tahun &
\multicolumn{1}{c|}{Judul Artikel} &
\multicolumn{1}{c|}{Metode yang digunakan} &
\multicolumn{1}{c|}{Hasil yang diperoleh} &
\multicolumn{1}{c|}{Persamaan} &
\multicolumn{1}{c|}{Perbedaan} \\ \hline
	3 	& \makecell{Shena Star Sarwodi\\ Wibisono Sukmo Wardhono \\ Muhammad Aminul \\Akbar (2020)}
		 & Penerapan Multiplayer Pada Gim Tower Defense Menggunakan Photon Unity
		 & whitebox, fps, delay dan Game Experience Questioner (GEQ).
		 & Dengan menerapkan Photon Unity Networking pada gim tower defense maka dapat diimplementasikan sebuah fitur yang dapat meningkatkan interaktivitas dan ketertarikan pemain pada gim yaitu fitur multiplayer
		 &Persamaannya yaitu menggunakan unity photon.
		 &Perbedaannya terdapat pada metode pengujian.
		 \\ \hline
	4 	& \makecell{Ryan Nanda Pratama\\,  Anton Siswo Raharjo Ansori\\ Ashri Dinimaharawati\\(2021)}
		 & PEMBUATAN MULTIPLAYER GAME UCING BELING MENGGUNAKAN ASSET STORE MIRROR
		 & blacbox
		 & Pada multiplayer game Ucing Beling dapat dimainkan secara realtime dan berjalan dengan sesuai yang diharapkan
		 & Persamaannya sama sama multiplayer.
		 & Perbedaannya peniliti menggunakan asset store mirror sebagai multiplayer platform
		 \\ \hline		
	\end{tabular}
	\end{table}
	\end{landscape}

\section{Tinjauan Teoritis}
\subsection{Unity}
Unity merupakan salah satu game engine paling populer saat ini. Unity adalah sebuah software development yang terintegrasi untuk menciptakan video game atau konten lainnya seperti visualisasi arsitektur atau real-timeanimasi baik yang bernuansa 2D maupun 3D. Unity juga merupakan perangkat lunak yang digunakan untuk mengembangkan game multiplatform yang didesain secara user friendly 
(Iman, 2017). Keunggulan Unity adalah Unity 
dapat dengan mudah mengontrol objek-objek 
dalam gim atau aplikasi. Unity terdapat 2 jenis 
lisensi yaitu personal edition yang dapat diakses 
secara gratis dan professional edition yang 
diharuskan untuk membayar perbulan untuk 
mengaksesnya dengan beberapa fitur tambahan 
yang tidak terdapat di personal edition.