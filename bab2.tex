\begin{landscape}
	\chapter{TINJAUAN PUSTAKA}
	\section{\textit{State of the Art}}
	\noindent

	\textit{State of the Art} Dalam penyusunan penilitian ini, peniliti mengambil beberapa referensi terdahulu sebagai panduan penulis untuk penilitian yang dilakukan, yang kemudian  akan menjadi acuan dan perbedaan dari penilitian yang akan dilakukan dengan penilitian sebelumnya. Pemaparan \textit{State of the Art} dapat dilihat pada tabel \ref{tb:stateoftheart} berikut.
	
	\begin{center}
	\begin{longtable}{| c | L{3cm} | L{4cm} | L{2.5cm} | L{4cm} | L{3cm} | L{3cm} |}
	\caption{Paparan \textit{State of the Art}}
	\label{tb:stateoftheart} \\
	
	\hline 
	No &
	Penulis/Tahun &
	\multicolumn{1}{c|}{Judul Artikel} &
	\multicolumn{1}{c|}{Metode yang digunakan} &
	\multicolumn{1}{c|}{Hasil yang diperoleh} &
	\multicolumn{1}{c|}{Persamaan} &
	\multicolumn{1}{c|}{Perbedaan} \\ \hline
	\endfirsthead
	
	\hline 
	No &
	Penulis/Tahun &
	\multicolumn{1}{c|}{Judul Artikel} &
	\multicolumn{1}{c|}{Metode yang digunakan} &
	\multicolumn{1}{c|}{Hasil yang diperoleh} &
	\multicolumn{1}{c|}{Persamaan} &
	\multicolumn{1}{c|}{Perbedaan} \\ \hline
	\endhead
	
	\hline \multicolumn{7}{|r|}{{Bersambung}} \\ \hline
	\endfoot
	
	\hline \hline
	\endlastfoot
	
	1 	& IKROM AULIA FAHDI (2016)
		 & ASYNCHRONOUS \textit{MULTIPLAYER} CARD \textit{GAME} PADA \textit{GAME} MONSTER KING
		 & blacbox
		 & Mekanisme asynchronous berhasil diimplementasikan menggunakan Photon Realtime.
		 & Persamaannya yaitu menggunakan unity photon
		 & Perbedaannya terdapat pada framework yang digunakan peniliti terdahulu menggunakan asynchronous.
		 \\ \hline
	2 	& Ibnu Ramadhan, Agung Purwanto dan Nurahman (2020)
		& PENGEMBANGAN TEKNOLOGI \textit{GAME} INDONESIA UNTUK PERMAINAN \textit{FIRST PERSON SHOOTER} (FPS) 3D \textit{MULTIPLAYER} “CODE TO SHOOT” MENGGUNAKAN UNITY NETWORK (UNET) BERBASIS MOBILE
		& blacbox
		& \textit{Game} ini dapat dimainkan secara \textit{multiplayer} tanpa perlu memasukan alamat IP karena fitur uNet dapat bekerja dengan baik. Selain itu \textit{game} ini juga sudah dapat dimainkan menggunakan platform mobile (Android)
		& Sama sama \textit{game} bergenre \textit{first person shooter}(fps)
		& Perbedaannya peniliti menggunakan unet sebagai \textit{multiplayer} platform
		\\ \hline
	3 	& Shena Star Sarwodi, Wibisono Sukmo Wardhono, Muhammad Aminul Akbar (2020)
		& Penerapan \textit{Multiplayer} Pada Gim Tower Defense Menggunakan Photon Unity
		& whitebox, fps, delay dan \textit{Game} Experience Questioner (GEQ).
		& Dengan menerapkan Photon Unity Networking pada gim tower defense maka dapat diimplementasikan sebuah fitur yang dapat meningkatkan interaktivitas dan ketertarikan pemain pada gim yaitu fitur \textit{multiplayer}
		&Persamaannya yaitu menggunakan unity photon.
		&Perbedaannya terdapat pada metode pengujian.
		\\ \hline
	4 	& Ryan Nanda Pratama,  Anton Siswo Raharjo Ansori, Ashri Dinimaharawati (2021)
		& PEMBUATAN \textit{MULTIPLAYER} \textit{GAME} UCING BELING MENGGUNAKAN ASSET STORE MIRROR
		& blacbox
		& Pada \textit{multiplayer} \textit{game} Ucing Beling dapat dimainkan secara realtime dan berjalan dengan sesuai yang diharapkan
		& Persamaannya sama sama \textit{multiplayer}.
		& Perbedaannya peniliti menggunakan asset store mirror sebagai \textit{multiplayer} platform
		\\ \hline	
	5 	& Mukhtar Halim (2013)
		& PEMBUATAN \textit{GAME} “THE LAST MISSION” DENGAN MENGGUNAKAN FPS CREATOR 
		& blacbox
		& Pembuatan \textit{game} berjenis FPS(\textit{First Person Shooter}) menggunakan FPS Creator 
		Free dapat meminimalisir kebutuhan sumber daya yang dibutuhkan dalam 
		pembuatan \textit{game}.
		&Persamaannya yaitu sama sama \textit{game} fps.
		&Perbedaannya terdapat pada \textit{game} engine yang digunakan yaitu FPS Creator
		\\ \hline
			  
	\end{longtable}
	\end{center}
	\end{landscape}

\section{Tinjauan Teoritis}
\subsection{Unity}
\noindent

Unity merupakan salah satu \textit{game} engine paling populer saat ini. Penggunaan Unity dapat digunakan untuk mengembangkan konten interaktif seperti video \textit{game}, 
visualisasi arsitektur, dan real-time 3D animasi. Unity menggunakan bahasa pemograman JavaScript dan 
C\# \cite{Ansori}. Unity juga merupakan perangkat lunak yang digunakan untuk mengembangkan \textit{game} \textit{multiplatform} yang didesain secara user \textit{friendly} 
(Iman, 2017). Keunggulan Unity adalah Unity 
dapat dengan mudah mengontrol objek-objek 
dalam gim atau aplikasi. Unity terdapat 2 jenis 
lisensi yaitu \textit{personal edition} yang dapat diakses 
secara gratis dan \textit{professional edition} yang 
diharuskan untuk membayar perbulan untuk 
mengaksesnya dengan beberapa fitur tambahan 
yang tidak terdapat di \textit{personal edition} \cite{Sarwodi}. 

\subsection{\textit{Multiplayer}}
\noindent

\textit{Multiplayer} merupakan fitur pada \textit{game} dimana pemain bermain dengan lebih dari 1 orang yang bermain 
di lingkungan \textit{game} yang sama dan waktu yang bersamaan. \textit{Game} \textit{Multiplayer} biasanya memberikan pilihan pada 
pemain untuk berbagi sumber daya sistem \textit{game} atau menggunakan internet untuk bermain bersama dalam jarak 
jauh. \textit{Game} \textit{Multiplayer} yang terhubung dengan internet melibatkan pemain yang saling terhubung melalui server. 
Sedangkan \textit{Game} \textit{Multiplayer} dengan koneksi lokal yaitu, pemain saling terhubung secara langsung dengan 
pemain lainnya, pemain terkoneksi menggunakan jaringan peer to peer. Pada \textit{Game} \textit{Multiplayer} online memiliki 
beberapa jenis kategori diantaranya adalah \textit{Massively} \textit{Multiplayer} Online \textit{game} (MMO), \textit{Massively} \textit{Multiplayer} 
Online \textit{First-person Shooter} \textit{Game} (MMOFPS), \textit{Massively} \textit{Multiplayer} online \textit{Real-time Strategy} \textit{Game}
(MMORTS), \textit{Massively} \textit{Multiplayer} Online Role-playing \textit{Game}s (MMORPG), \textit{Multiplayer} Online Battle Arena
(MOBA)\cite{Ansori}.

\subsection{Photon Unity Networking (PUN)}
\noindent

Photon adalah sebuah framework pengembangan \textit{game} \textit{multiplayer} \textit{real-time} yang cepat, ringan, dan fleksibel. Photon terdiri dari server dan beberapa SDK klien untuk platform utama.
Photon Unity Network (PUN) adalah solusi khusus Unity yang dihadirkan dengan tingkat yang lebih tinggi: matchmaking, panggilan balik yang mudah digunakan, komponen untuk sinkronisasi \textit{Game}Objects, Remote Procedure Calls (RPCs), dan fitur serupa yang memberikan awal yang baik. Di luar itu, terdapat API yang solid dan luas untuk kontrol yang lebih canggih \cite{pun}.

\subsection{\textit{First Person Shooter}(FPS)}
\noindent

\textit{First Person Shooter} (FPS) adalah salah satu jenis \textit{game} yang saat ini sangat digemari terutama kalangan \textit{game}rs muda. FPS merupakan \textit{game} yang menggunakan sudut pandang orang pertama dimana pemain akan dibuat seolah-olah menjadi karakter utama dalam \textit{game} dengan tampilan yang berpusat pada permainan disekitar senjata atau alat yang sedang digunakan \cite{fps}.

\textit{First person shooter} merupakan jenis 3D \textit{game} shooter yang menampilkan sudut pandang orang pertama dengan 
pemain yang melihat aksi melalui mata karakter permain. Tidak seperti orang ketiga yang terlihat dari bagian 
belakang atau samping, yang memungkinkan \textit{game}r untuk melihat karakter secara keseluruhan\cite{fps}.

FPS dikembangkan pada tahun 1973 melalui permainan ruang yang belum sempurna yaitu flight simulator, yang 
menampilkan sudut pandang orang pertama dengan mengarah lebih rinci ke simulator pesawat tempur, dikembangkan untuk pasukan AS pada akhir tahun 1970-an. Permainan ini tidak lagi tersedia untuk konsumen \cite{fps}.

\subsection{C\#}
\noindent

C\# (C-sharp) adalah salah satu bahasa pemograman yang menggunakan Framework .NET. Sama seperti 
bahasa lainnya, C\# memiliki aturan pada syntax dan kode-kode yang bisa digunakan dalam pembuatan aplikasi. 
C\# cocok untuk dipelajari untuk pemula karena aturan syntax-nya lebih sederhana dibandingkan bahasa 
pemograman lainnya \cite{Ansori}.

\subsection{\textit{Game} Sosial}
\noindent

\textit{\textit{Game}} sosial merupakan \textit{\textit{game}} yang dapat beraktifitas dan 
berinteraksi secara \textit{virtual} antar sesama penggunanya. Cara 
mempertemukan penggunanya secara umum ada dua metode, 
yaitu dengan metode \textit{synchronous} dan \textit{asynchronous}. Pada 
metode \textit{synchronous}, antar pengguna dipertemukan secara 
realtime ketika bermain sedangkan pada metode \textit{asynchronous}, 
antar pengguna dapat berinteraksi tanpa dibatasi oleh waktu \cite{asyncyuhu}


