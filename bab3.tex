\chapter{METODOLOGI PENELITIAN}
Metodologi penelitian menggambarkan urutan langkah pelaksanaan penelitian atau strategi peneliti, rencana, tempat, waktu, pengambilan data dalam menjawab masalah penelitian. Pada bagian ini diuraikan metode yang digunakan secara rinci. Dengan demikian dapat diperkirakan hasil penelitian yang akan diperoleh secara utuh. Dalam bagian metodologi penelitian perlu diuraikan beberapa hal berikut:

\section{Data dan Pengumpulan Data}
Pada bagian ini diuraikan jenis data digunakan  (data primer atau skunder). Data primer adalah data yang diperoleh  peneliti melalui pengukuran langsung dan bukan bersal dari data yang telah ada, sedangkan data skunder adalah data yang dikumpulkan oleh pihak tertentu dan telah didokumentasikan sehingga dapat digunakan oleh pihak lain yang membutuhkannya. Dalam suatu penelitian  memungkinkan peneliti menggunakan  jenis  data primer saja atau  data skunder saja, atau kedua-duanya( data primer dan skunder).  Di bidang keteknikan, data sekunder biasanya diperlukan peneliti untuk pembuatan sebuah aplikasi atau sistem pada lokasi tertentu. Berkaitan dengan pengumpulan data, harus diuraikan cara pemerolehan data, baik data primer, maupun data sekunder. Cara yang dimaksud sering disebut dengan teknik pengumpulan data. Ada beberapa teknik pengumpulan data yaitu : Observasi, Wawancara dan questioner.

\section{Rancangan Sistem(software/hardware)}
Bagian Rancangan system ini anda harus uraikan secara terperinci  proses rancangan   sistem/pengukuran/pengambilan data yang akan dilakukan. Rancangan sistem yang dibuat dapat berupa Unified Modelling Language yang memuat Usecase dan diagram dan activity diagram atau Context Diagram dan Diagram Level 1 atau  Blok diagram sistem yang dirancang ataupun arsitektur sistem yang akan dibuat. Rancangan system yang dipilih harus diuraikan prinsip kerjanya secara rinci agar dapat dipahami dengan mudah dan jelas. Jika penelitian yang akan dilakukan  berkaitan dengan pengukuran, harus dicantumkan set-up pengukuran dan hasil pengukuran. Selain itu, perlu disebutkan juga parameter-parameter yang dominan dalam rancangan/pengukuran.
Sebagai penjelasannya bahwa pada rancangan sistem yang dibuat hendaknya memuat Rancangan INPUT, PROSES dan OUTPUT. 
Selain itu itu jika rancangannya berupa UML atau Context Diagram maka diagram yang dibuat memuat setidaknya ada requirement data, Requirement fungsional system, dan requirement non fungsional system

\section{Metode dan Variabel Penelitian}
Pada bagian ini diuraikan pembuatan/implementasi sistem,   metode/algoritma yang digunakan, jika memungkinkan disertai dengan flowchart dari proses implementasi dan algoritma yang digunakan.

\section{Teknik Pengujian}
Pada bagian ini diuraikan tentang keterkaitan antarfaktor dari data yang diperoleh  berdasarkan masalah yang dirumuskan pada bagian rumusan masalah. Selanjutnya,  masalah tersebut diselesaikan dengan metode/algoritma yang digunakan, dianalisis proses dan hasil penyelesaian masalah tersebut.  Dalam hal ini, perlu disebutkan parameter-parameter yang akan diuji dan dianalisis pada penelitian yang dimaksud beserta dengan alat ujinya.

\section{Hasil yang diharapkan}
Pada bagian ini dicantumkan prediksi hasil penelitian. Dalam hal ini, dapat dipaparkan kira-kira hasil akhir penelitian yang akan dicapai berupa apa?