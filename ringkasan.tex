\chapter*{RINGKASAN}
\addcontentsline{toc}{chapter}{RINGKASAN}
Kemajuan dalam perkembangan teknologi game yang setiap tahunnya berkembang di 
software dan hardware khususnya  perangkat mobile yang mendukung sistem operasi Android 
kini mampu menjalankan aplikasi game dalam skala 3D dan mengintegrasikan software dan hardware yang dihasilkan. 
Kinerja optimal Pengalaman bermain game pengguna sangat bervariasi tergantung pada cara  meningkatkan pengalaman bermain game untuk pengguna. 
Memainkan secara bersamaan dengan pemain lain adalah salah satu cara untuk membuat pengalaman bermain lebih menyenangkan. 
Skripsi ini membangun game bergenre fps dengan gameplay  interaktif untuk 
karakter yang dapat dikontrol dan mengimplementasikan \textit{synchronous multiplayer} dalam game tersebut. 
Pemain dapat langsung berinteraksi dengan 
pemain selama bermain \textit{game} karena bertemu secara real time. 
Metode yang dimplementasikan pada skripsi ini adalah kombinasi game engine unity dan \textit{framework}-nya sendiri yaitu \textit{photon unity networking(pun)}
dengan fungsi untuk menyatukan pemain secara realtime untuk menciptakan suasana yang menyenangkan dalam \textit{game}.
\newline \break
\noindent Kata Kunci: FPS Multiplayer, Online Multiplayer, Photon Unity Networking.