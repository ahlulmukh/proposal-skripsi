\chapter*{RINGKASAN}
\addcontentsline{toc}{chapter}{RINGKASAN}
Kemajuan perkembangan teknologi \textit{game} yang tiap 
tahunnya berkembang pada perangkat lunak dan perangkat keras 
khususnya pada perangkat \textit{mobile} yang mendukung sistem 
operasi Android, kini telah mumpuni menjalankan aplikasi game 
berskala 3D dan pengintegrasian antar perangkat lunak dan 
perangkat keras yang menghasilkan performa yang optimal. 
Pengalaman bermain pengguna sangat bervariasi dari 
segi aspek yang berbeda-beda cara untuk meningkatkan 
pengalaman bermain pada pengguna. Bermain dalam satu waktu 
yang sama dalam sebuah permainan dengan pemain lainnya 
merupakan salah satu caranya untuk menciptakan pengalaman 
bermain \textit{game} menjadi lebih menyenangkan. 
Dalam Skripsi ini dibangun sebuah permainan yang 
bergenre fps yang memiliki \textit{gameplay} yang interaktif untuk 
mengendalikan karakter dan menerapkan tipe permainan 
\textit{synchronous} \textit{multiplayer} untuk mode online dalam sebuah 
permainan. Antar pemain dapat langsung berinteraksi ketika 
bermain karena dipertemukan secara \textit{realtime}. Metode 
penerapan untuk merealisasikannya tersebut akan menggunakan 
perpaduan dari unity engine dengan framework photon unity 
network untuk mengintegrasikan unity engine dengan photon 
cloud. Dengan fungsionalitas aplikasi permainan tersebut 
diketahui bahwa cara bermain yang mempertemukan antar 
pemain secara \textit{realtime} akan menciptakan suasana 
menyenangkan dalam bermain.

\noindent \textbf{Kata Kunci: FPS Multiplayer, Unity, Photon Unity, Photon Cloud, Photon Unity Networking.}